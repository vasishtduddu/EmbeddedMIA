\documentclass[sigconf]{acmart}
%\documentclass[conference]{IEEEtran}

\settopmatter{printacmref=false} % Removes citation information below abstract
\renewcommand\footnotetextcopyrightpermission[1]{} % removes footnote with conference information in first column
\pagestyle{plain} % removes running headers

\usepackage[T1]{fontenc}
\usepackage{algorithm}
\usepackage{algorithmic}
\usepackage{multirow}
\usepackage{colortbl}
\usepackage{dblfloatfix}
\usepackage{pifont}
\usepackage{tikz}
\usepackage{pgfplots}
\usepackage{enumitem}
\usepackage{wasysym}
\usetikzlibrary{backgrounds, positioning, fit}
\usetikzlibrary{shapes.geometric}
\usepackage{amsmath}
\usetikzlibrary{patterns}
\usetikzlibrary{pgfplots.groupplots}
\newcommand{\ballnumber}[1]{\tikz[baseline=(myanchor.base)] \node[circle,fill=.,inner sep=1pt] (myanchor) {\color{-.}\bfseries\footnotesize #1};}
\newcommand{\method}{{\scshape Gecko}}
\newcommand{\cmark}{\CIRCLE}
\newcommand{\xmark}{\Circle}
\newcommand{\smark}{\LEFTcircle}


\begin{filecontents}{comparedef.txt}
z   n   pFA pFB
10  FP   98.16   86.16
10  AdvReg   87.48   83.66
10  DP   92.80   85.11
10  Gecko   90.49   83.52
20  {}  0   0
20  FP   97.86   80.34
20  AdvReg   73.97  71.02
20  DP   84.76  79.27
20  Gecko   76.79  73.5
30  {}  0   0
30  DP   99.58  88.95
30  AdvReg   89.09  85.19
30  DP   90.46  84.91
30  Gecko   93.45  85.8
\end{filecontents}

\begin{document}

\title{\method: Reconciling Privacy, Accuracy and Efficiency in Embedded Deep Learning}

\author{Vasisht Duddu$^1$, Virat Shejwalkar$^2$, Antoine Boutet$^1$}
\affiliation{\institution{$^1$ \large Univ Lyon, INSA Lyon, Inria, CITI \\ $^2$ \large University of Massachusetts Amherst}}
\email{vduddu@tutamail.com, vshejwalkar@cs.umass.edu, antoine.boutet@insa-lyon.fr}


\begin{abstract}
Embedded systems demand on-device processing of data using Neural Networks while conforming to the memory, power and computation constraints.
In order to bring the high performing Neural Networks to edge devices, several state of the art optimizations such as model compression through \textit{pruning}, \textit{quantization}, and careful \textit{design of efficient architectures} have been extensively adopted.
These algorithms when deployed to real world sensitive applications, such as healthcare monitoring wearables, requires to resist inference attacks to protect privacy of user's training data.
We quantify this privacy leakage using membership inference attacks where the adversary aims to infer whether a given data point was a member of the model's training data or not.
In this work, we address the three-dimensional \textit{privacy-accuracy-efficiency} tradeoff in Neural Networks for embedded systems and propose \method\hspace{0.02in} training methodology where we explicitly add efficiency of private inference as a design objective of Neural Networks.
We use the inference-time memory, computation and power constraints of embedded devices as a criterion for designing Neural Network architecture while preserving membership privacy.
Given the flexibility of modifying a model during training, we choose quantization as our design choice for highly efficient and private models.
This choice is driven by the observation that compressed models leaks more information compared to baseline models while off-the-shelf efficient architecture designs indicate poor efficiency and privacy trade-off.
We show that models trained using \method\hspace{0.02in} methodology are comparable to prior state of the art defences against blackbox membership inference attacks in terms of test accuracy and privacy while additionally providing efficiency.
\end{abstract}
\keywords{Membership Privacy, Inference Attacks, Efficient Deep Learning, Embedded Computing.}

\maketitle


\section{Introduction}\label{introduction}

The tremendous performance of Machine Learning, especially Deep Learning, models has resulted in their deployment to low-powered edge devices and embedded systems.
Specifically, Internet of Things (IoT) devices extensively prefer on-device processing to reduce communication latency and overhead, while also preserving the privacy of data from an untrusted data curator~\cite{8110880}.
The design of efficient Neural Networks (NNs) requires algorithm-hardware co-design such as model compression, quantization, and designing special architectures with higher efficiency~\cite{8114708}.
% that are extensively used for industry applications
Such NN architecture design optimizations should conform to efficiency constraints on memory, energy, and computation overhead on embedded devices, and also maintain high prediction accuracy.
However, such designs often result in \textit{efficiency-accuracy} trade-off~\cite{rastegari2016xnornet}.

Additionally, privacy laws, such as HIPAA and GDPR, require on-device processing to maintain the privacy of user's sensitive data (e.g, medical records, location traces, and purchase preferences).
In this work we focus on Membership Inference Attack~\cite{shokri2017membership}, where given a target model and a target record, the adversary determines if the target data record was part of the target model's training data by analyzing the target model's output predictions.
For instance, wearable devices, which monitor its user's health, commonly rely on NNs for various health related predicts. Such devices are continuously trained on the private data of a large number of users, and therefore, by mounting membership inference attacks on target device, an adversary can determine if the data of a target user was used to train NNs on the target device.
In such cases, it is crucial to design NNs resistant to inference attacks, where the adversary infers unobservable, sensitive information (e.g, user's health status) from the observable information (e.g., model predictions).
We refer to computations that achieve privacy through inference-resistance as {\em privacy-preserving computation}.
%Multiple works propose privacy preserving mechanisms to address the issue of privacy leakage
Such privacy preserving computation mechanisms affect the model's predictive accuracy resulting in \textit{privacy-accuracy} trade-off~\cite{Abadi:2016:DLD:2976749.2978318,DBLP:conf/ccs/NasrSH18,shejwalkar2019reconciling}.

Considering the trade-offs described above, the three objectives to consider while designing NNs for embedded devices are: (a) high prediction accuracy, (b) efficiency constraints on memory, energy, and computation overhead, and (c) preserving privacy of on-device data.
However, designing a model to preserve privacy while satisfying efficiency requirements without a significant cost of the model’s predictive accuracy is challenging.
%Designing NNs that meet all of the above objectives is a significantly challenging task due to the aforementioned trade-offs.
%In this work, we address the research question: \textit{How can we redesign NNs to reconcile privacy and efficiency in Deep NNs without a significant loss in accuracy?}

In this paper, we address this challenge by proposing \method\hspace{0.02in} — a two phase training methodology for designing NNs optimized specifically for performance, accuracy and privacy. We evaluate the privacy leakage of three state of the art hardware software co-design techniques, namely, model compression, quantization and efficient off-the shelf architectures. We show that model compression leaks more information compared to baseline (uncompressed) models indicating a higher privacy risk to the user’s data while off-the-shelf architectures (MobileNet and SqueezeNet) do not meet all the efficiency requirements but can provide limited privacy leakage. These observations motivate our design choice of quantizing NNs as part of \method\hspace{0.02in} training algorithm.

%In this paper, we first highlight the three-way trade-off between accuracy, efficiency, and privacy due to NNs designed for embedded devices.
%More specifically, we evaluate membership privacy leakage due to the three state-of-the-art hardware-software co-design techniques: model compression, quantization, and efficient off-the shelf architectures.
%We show that model compression leaks more information compared to the baseline (uncompressed) models and poses higher privacy risks to the user's data.\virat{add some numbers here}
%\red{Off-the-shelf architectures (e.g., MobileNet and SqueezeNet) do not meet all the efficiency requirements but can provide more limited privacy leakage.}\virat{clarify using numbers}
%These observations motivate our design choice of quantizing NNs as part of \method\hspace{0.02in} training algorithm.
%It is worth noting that, previous works have explored the the efficiency-accuracy~\cite{rastegari2016xnornet} and privacy-accuracy~\cite{shokri2017membership} trade-offs, however separately. To address the above shortcoming, our work provides the first systematic evaluation efficiency-accuracy-privacy trade-offs and leverages the lesson learned from the evaluation to design a novel training methodology.

% \noindent\textbf{Our contributions:} In this paper, we address this challenge by proposing %provide directions to design NNs for efficiency private inference.
% %We present 
% \method~ --- a two phase training methodology to design NNs optimized specifically for accuracy, efficiency, and privacy.
% We evaluate the privacy leakage of three state of the art hardware software co-design techniques, namely, model compression, quantization and efficient off-the shelf architectures.
% We show that model compression leaks more information compared to baseline (uncompressed) models indicating a higher privacy risk to the user's data while off-the-shelf architectures (MobileNet and SqueezeNet) do not meet all the efficiency requirements but can provide more limited privacy leakage.
% %acceptable 
% % this term should be avoided (what is acceptable?)
% These observations motivate our design choice of quantizing NNs as part of \method\hspace{0.02in} training algorithm.

%we propose \method\textemdash a two phased training methodology to design NNs optimized specifically for accuracy, efficiency, and privacy. 
In Phase-I of \method, the model parameters and activations are binarized, i.e., constrained to \{-1,+1\} to reduce the memory, energy consumption, and computation overhead.
To ensure computation efficiency, we replace the expensive multiply accumulate operations between parameter matrices and activation vectors 
% from previous layers 
to simple and cheaper XNOR operations.
As our main contribution, we show that aggressively quantized NN architectures obtained in Phase-I ensure efficient privacy-preserving computation with higher resistance to membership inference attacks.%\red{why is it better?}
Phase-I of \method\hspace{0.02in} optimizes for efficiency \textit{and} privacy but at the cost of a significant drop in accuracy.
In Phase-II, we restore this accuracy 
%to acceptable standard compared to the original (full precision model) accuracy
by transferring knowledge from larger full precision models to the quantized models~\cite{shejwalkar2019reconciling}.
Here, the quantized XNOR model uses the output predictions of the full precision state of the art models as labels instead of using the true labels during training.
This results in significantly increase in the prediction accuracy of the model while limiting privacy leakage.
%with a small but acceptable increase in privacy leakage.

%In summary, \method\hspace{0.02in} training algorithm comprises of Phase I of \textit{Quantizing} to reduce the precision of model with binary values with XNOR computations which provides efficiency and privacy.
%This is followed by Phase II of \textit{Distillation} to restore the accuracy degradation by transferring the knowledge learnt from the full precision to the efficient quantized model.

Finally, we compare the models trained using \method\hspace{0.02in} with prior state-of-the-art defences against membership inference attacks, namely, Adversarial Regularization~\cite{DBLP:conf/ccs/NasrSH18} and Differential Privacy~\cite{Abadi:2016:DLD:2976749.2978318}.
We show that our proposed models improve the trade-offs between the efficiency,  accuracy, and privacy compared to the baselines approaches. The code is made publicly available\footnote{Anonymized for Submission}.
%, and show that our proposed models performance is comparable to the state of the art.
%However, our models provide an additional guarantee of higher efficiency which models trained with prior defences fail to provide.

The paper is outlined as follows: Section~\ref{background} presents background in embedded Deep Learning while Section~\ref{analysis} reports a comparative analysis of state of the art baselines.
We describe \method\hspace{0.02in} in Section~\ref{design} and evaluate it Section~\ref{compare}. Related work are then presented Section~\ref{related} before to conclude Section ~\ref{conclusions}.

%The paper is outlined as follows: We describe the threat model and describe the state of the art algorithms for embedded Deep Learning in Section~\ref{background} followed by identifying the best optimization technique to satisfy privacy and efficiency in Section~\ref{motivate}. Based on the observations, we describe the Phase I and Phase II design of \method\hspace{0.02in} training algorithm in Section~\ref{design}.
%Finally, we show that the proposed optimization for NNs is comparable to prior state of the art privacy preserving defences (Section~\ref{compare}) while additionally providing efficiency guarantees.

\section{\method: Framework Setting}\label{background}
\virat{i think this section should be called background}

\subsection{Machine Learning}
Machine Learning algorithms learn a function $f:X \rightarrow Y$ mapping from the input space $X$ to the space of corresponding class labels $Y$.
This is modelled as an optimization where the objective is to find the parameters $\theta$ by minimizing the model's loss, $min_{\theta} L(f(x),y;\theta)$, computed when model incorrectly predicts a given data input $x$.
\virat{define loss?}
The dataset sampled from the data distribution $P(X,Y)$ is aggregated from user's sensitive and personal information such as location history, purchase preferences, medical records and financial data.
\virat{here, keep everything technical; don't say where does the data come from.}

\subsection{Membership Inference Attacks}

% On querying a trained model $f(x;\theta)$, the adversary learns aggregate information about the entire data population which the model generalizes from the train dataset to an unseen test data.
% This is desirable and quantified using the accuracy of the model.
% On the other hand, if the adversary learns something specific about a user's data record used in the training dataset, we refer to such information as privacy leakage.
% In other words, in context of machine learning, there is a privacy breach if the adversary learns unobservable information specific to an individual user's data record from observable information such as model's output predictions.
% This inferred unobservable information about a user's record can be, for instance, the membership details of the record in the training set of the model, referred to as membership inference attacks.
% Alternatively, an adversary can learn sensitive attributes about the user's data record which can be used to reconstruct the sensitive training dataset.
% In this work, we specifically use membership inference attacks to quantify information leakage in machine learning models.

Machine learning (ML) models are generally trained for a specific task.
However, it is well known that ML models, especially the deep neural networks, tend to learn unintended information from their training data \cite{}.
If such training data is of sensitive nature, e.g., location traces or medical records, corresponding ML model may leak unintended information through their predictions.
Previous literature has demonstrated multiple attacks that leak privacy of training data~\cite{}.
Specifically in this work, we focus on \emph{Membership Inference Attack} (MIA).
Given a target data sample and target ML model, MIA aims to infer whether the target sample was a part of the training data of the target ML model.
\virat{give intuition in one to two short sentences}



\subsubsection{Threat model}\label{mia_threat_model}
We detail the threat model of MIA that we consider in this work.
% Machine learning models are more confident while predicting the class of already seen train data record compared to an unseen test data record.
% Membership inference attacks exploits this difference in the model's confidence to classify a new data record as being a "Member" or "Non-Member" of the model's training data.
% This is a binary decisional problem where the adversary classifies the membership of a given input $x$ using the model's output prediction $f(x;\theta)$.
% We describe the threat model and attack details below.


\noindent\textbf{Adversary goal.} \virat{make a pass and write the goal formally and concisely}
% The goal of adversary in membership inference attack is to infer whether a given data record was used in the model's training data or not.
Formally, given a user's data record $x$ $\sim$ $P(X,Y)$, where $P(X,Y)$ is the data distribution from which the training data $D_{train}$ was sampled, the adversary estimates $P(x \in D_{train})$ using the model's prediction $f(x;W)$.
\virat{no need of how adv attacks, just tell the goal}Empirically, the adversary identifies a threshold to estimate whether $x \in D_{train}$ which can also be learnt using a binary classifier.

\noindent\textbf{Adversary's capabilities.}
\virat{explain what can adv do: e.g., no tampering the training alg, training data, or received prediction}

\noindent\textbf{Adversary knowledge.} We assume a blackbox setting where the adversary is assumed to have no knowledge about the target model.
Formally, given a target model $f()$ which maps an input data record $x$ to its correct label $y$, the adversary only sees the final model prediction $f(x;\theta)$.
The adversary does not know the architecture of $f()$ and the model parameters $\theta$.
This is a practical setting seen typically in Machine Learning as a Service (MLaaS) where the adversary submits an input query to the trained model on the Cloud via an API and receives the corresponding output predictions.


\subsubsection{Attack Methodology}
\virat{detail this section a bit more as you use this throughout! Explain why is this enough, e.g. refer to papers that show whitebox attacks or shadow model attacks give accuracy close to thresholding if done properly}
In this work, we use the confidence score attack where the adversary leverages the prediction entropy of the target model $f(x;W)$ to perform the membership inference attack [][].
In particular, the adversary obtains $f(x;W)$ and finds the maximum posterior and infers $x \in D_{train}$ if the maximum is greater than a threshold.
The attack is based on the observation that the maximal posterior of a member data record is higher (more confident) than a non-member data record of the training dataset.



\subsection{Algorithms for Embedded Deep Learning}

On-device processing is an attractive alternative compared to centralized processing of data from sensors, IoT, and embedded edge devices.
Such on-device processing reduces the overhead of communicating data from the devices to the servers, lowers the privacy and security risk associated with storing sensitive data on untrusted central server and lowers the latency for obtaining results from processing~\cite{}.
However, in order to execute trained NN\virat{define NN if u haven't} models on low powered edge devices, additional optimization of the architectures of ML models is required.
Three state-of-the-art approaches to design efficient NN models for embedded systems are: (1) model compression via pruning, (2) quantization of model parameters and activations and (3) designing standard architectures (off the shelf efficient architectures).
In this work, we consider these three approaches as baselines for comparison and evaluation to select the choice of optimization for \method.\virat{i don't get this sentence; what do u mean?}

\noindent\textbf{Model Compression (Pruning).} NNs have large number of redundant weights~\cite{}.
Pruning removes redundant weights, i.e., sets them to zero, without degradation of model accuracy.
A pruned model has sparse parameters and a hardware can be designed to skip the multiplication\virat{why skip multiplications? are they dim-wise?} and memory storage, and improve the efficiency.
Sparse weights\virat{use weights or parameters, but be consistent} can be stored in a compressed format in the hardware using the compressed sparse row or column format which reduces the overall memory bandwidth~\cite{}.
Aggressive pruning compresses the model significantly, but requires re-training to restore the model's original accuracy.
For a threshold $T$, the parameters close to zero are replaced by zero which is given as:
\[
    f(W)=
\begin{cases}
    0, & \text{if } -T \geq w \leq T\\
    w,  & \text{otherwise}
\end{cases}
\]

\noindent\textbf{Off-the-Shelf Efficient Architectures.} NNs can be redesigned by changing the hyperparameters, e.g., filter sizes, number of layers, etc., to reduce the number of parameters and hence, the memory footprint\virat{what is memory footprint? define or don't use jargon}.
For instance, large convolution filters can be replaced with multiple smaller filters with less number of parameters, while covering the same receptive field.
For instance, one 5x5 filter can be replaced by two 3x3 filters.
Alternatively, 1x1 convolutional layers reduce the number of channels in output feature map, which lowers the computation and the number of parameters.
% For instance, for an input activation of dimension 1x1x64, 32 1x1 convolutional filters downsamples the activation maps to get an output of 32 channels.
Such optimizations lead to compact NN architectures with smaller number of parameters compared to the architecture.
These have been extensively adopted to design standard architectures\virat{what's a standard architecture?} such as MobileNet~\cite{conf/cvpr/SandlerHZZC18} and SqueezeNet~\cite{DBLP:journals/corr/IandolaMAHDK16}.


\noindent\textbf{Quantization.}
% Quantization reduces the precision of the model's parameters and the intermediate activations during execution.
Quantization reduces precision of model parameters and activations and maps their values to a fixed set of quantization levels~\cite{Hubara:2017:QNN:3122009.3242044}.
The number of quantized levels determines the precision of the operands ($log_2(\#levels)$).
Reducing the precision of the (a) parameters lowers memory requirement of model, (b) activations lowers the computation overhead by replacing MACs\virat{what's MACs?} with binary arithmetic and (c) lowers the energy consumption by lowering the memory accesses and increasing throughput.\virat{something wrong with (c)'s starting}
Aggressively quantizing the parameters and activations to binary and ternary precision significantly improves the overall efficiency, however, at the cost of accuracy[Xnor][].
For instance, binarized\virat{binary?} NNs quantize the operands to \{-1,+1\} values[][]\virat{do u use [] to show empty cites? use the cmd cite{}} while ternary NNs have values \{-w, 0, w\} where $w$ can be fixed or learnt during training[].
The additional sparsity in ternary NNs reduce computation and storage cost. These are examples of uniform quantization.
Alternatively, weight sharing maps several parameters to a single value reducing the number of unique parameters~\cite{}.
This mapping is done using K-Means clustering or a hashing function and the corresponding shared values are read from a "codebook" which maps different parameters to its shared value.



\subsection{Datasets and Architectures}
\virat{this section should be in exp set up.}


For evaluating and comparing different efficiency algorithms, we use three simple datasets: FashionMNIST, Purchase100 and Location.
These provide us with the necessary direction to choose the optimal efficiency algorithm which satisfies all the efficiency and privacy requirement as describes in Section~\ref{motivate}.

\noindent\textbf{FashionMNIST.} The FashionMNIST dataset consists of 60,000 training examples and a test set of 10,000 examples.
Each data record is a 28$\times$28 grayscale image which is mapped to one of 10 classes consisting of fashion products such as coat, sneaker, shirt, shoes.
For FashionMNIST dataset, we use a modified LeNet architecture with two convolution layers followed by maxpool and dense layers: [Conv 32 (3,3), Conv 64 (3,3), Maxpool (2,2), Dense 128, Dense 10].

\noindent\textbf{Purchase100.} The  Purchase100  dataset  is a privacy sensitive dataset capturing the purchase preferences of online customers.
The pre-processed dataset is taken from the authors of~\cite{} which has been taken from the Kaggle's "acquired valued shopper" challenge\footnote{https://kaggle.com/c/acquire-valued-shoppers-challenge/data}.
The data records have 600 binary features and each record is classified into one of 100 classes identifying each user's purchase.
For Purchase100 dataset, we use a fully connected architecture with the nodes in each layers as [1024,512,256,128,100].

\noindent\textbf{Location.} The Location dataset is a privacy sensitive dataset capturing user's location "check-ins" in Bangkok collected from the Foursquare social network\footnote{https://sites.google.com/site/yangdingqi/home/foursquare-dataset} from April 2012 to September 2013.
We use the pre-processed dataset from the authors of~\cite{} where each record has 446 binary features which is mapped to one of 30 classes each representing a different geosocial location type (e.g., Restaurant, fast food joint, etc.). We use 1,600 data records to train the model.
For Location dataset we use a fully connected architecture with hyperparameters as [512,256,128,30].

However, the actual \method\hspace{0.02in} NN design methodology is evaluated on more sophisticated datasets such as CIFAR10.
Further, the dataset has been commonly used for evaluating defences against membership inference attacks, it enables to accurately compare our work with prior state of the art defences[][][].
The optimization described as part of \method\hspace{0.02in} is for large convolutional NNs and does not cover the fully connected dense layers (detailed description in Section~\ref{}) which are used for Purchase100 and Location datasets.
It is important to evaluate on standard architectures as different custom classifiers tend to underestimate the inference leakage due to hyperparameter settings.

\noindent\textbf{CIFAR10.} The CIFAR10 dataset is a major image classification benchmarking dataset where the data records are composed of 32$\times$32 RGB images where each record is mapped to one of 10 classes of common objects such as airplane, bird, cat, dog.
For CIFAR10 dataset, we use standard state of the art architectures: Network in Network (NiN), AlexNet and VGGNet.

For Location and Purchase datasets, we train the model for 50 epochs while for FashionMNIST dataset we train the model for 75 epochs; for CIFAR10 we train the models using standard hyperparameter setting for 100-150 epochs.
We train all the model until convergence and do not specifically overfit the models.


\subsection{Metrics}\virat{this section should be in exp set up.}

We use the inference attack accuracy to estimate the success of membership inference attack.
An accuracy above random guess $50\%$ indicates a training data leakage through membership inference attack.
This indicates that the adversary is able to identify the membership details of a data record with an accuracy higher than random guess.
The success of inference attack accuracy is strongly correlated with the model's extent of overfitting empirically measured as the difference between the train and test accuracy.
Additionally, the accuracy of the model is computed using the model's performance on unseen test data.

\section{\method: Design Motivation}\label{motivate}

In this section, we describe the requirements to be satisfied while designing NNs for embedded systems within the \method\hspace{0.02in} methodology.
We motivate the NN design by comparing different efficiency optimization algorithms based on the privacy and efficiency requirements of embedded devices.
In this work, we specifically select the three state of the art design techniques for efficient model computation:\virat{didnt u mention this above? if yes, just refer to them don't rewrite}
\begin{itemize}
\item Model Compression via pruning redundant parameters and nodes
\item Quantization to lower the precision of model parameters and activations
\item Efficient off-the-shelf architectures
\end{itemize}

\virat{I don't understand the flow of this section: You state 3 requirements, but why only one is detailed? Why 3.4 is evaluation? I think you should state the overall design requirements at first, then explain each requirement in detail (why only 'efficiency' is detailed?)}

\virat{I think u need to move 3.4 up that shows the leakage, then discuss the requirements, then introduce gecko and show how it addresses the shortcomings you found in 3.4 and also the other requirements u discuss}

\subsection{Overall Design Requirements}

In designing our training methodology, \method, we aim to satisfy the following main requirements in the purview of embedded systems:

\begin{itemize}[leftmargin=*]

\item {\em Privacy-}
The model should preserve the privacy of an individual's data record in the training set of the model against inference attacks.

\item {\em Efficiency-}
The private model should demonstrate high energy, memory and computation efficiency.

\item {\em Accuracy-}
The drop in accuracy of the private model should be minimum as compared to the non-private and non-efficient model accuracy.
\end{itemize}

To this end, we present \method --- a technique to construct NNs that are optimized for efficiency and accuracy while ensuring privacy of the input data.
With \method, we present a novel approach of building NNs with efficiency as a key property.
We argue that fixing a NN architecture and then modifying the training algorithm to ensure privacy (e.g. adversarial regularization[] and differential privacy[]) does not provide optimal balance between efficiency and accuracy.
On the contrary, our approach advocates building NN architecture considering the strengths and drawbacks of the underlying algorithms to design efficient NNs.
We assert that this is a practical solution as deep learning algorithms are flexible with respect to their architectures, i.e., different NNs can be trained to achieve the same accuracy for a given dataset.


\subsection{Efficiency Requirements}

In designing \method, we make several design choices with the goal of achieving efficiency over existing solutions.
The most important among them is the selection of the underlying algorithm to design NNs with high efficiency.
Several state of the algorithms are currently adopted such as, however, these algorithm do not provide the same efficiency guarantees.
These differences make it difficult to decide which primitive is the best fit for designing a privacy-preserving system for a particular application.
Therefore, we first outline the desirable properties specifically for private NN inference:

\begin{itemize}[leftmargin=*]
\item {\em Energy Efficiency-} Energy consumption is a vital constraint for low powered embedded or IoT devices which operate for long duration while maximising their battery lifetime.
While executing NNs, every MAC requires memory access for reading weights, inputs and intermediate output from previous layer and one write to store the computed outputl which is significantly higher than actually performing the MAC operation in the CPU[].
Energy efficiency is achieved by reducing the memory access by (a) optimizing hardware to exploit sparsity in MACs and (b) reducing the precision to increase the throughput of data.

\item {\em Computation Efficiency-} The total multiply accumulate (MAC) operations between the parameter matrix and input activation function quantifies the requirement of computation efficiency.
The processing rate of MAC operations is constrained by the CPU on embedded device which is reduced by reducing the total number of parameters.
Additionally, replacing MACs with cheaper binary arithmetic significantly lowers the computational overhead.

\item {\em Memory Efficiency-} The total size of the model measured in terms of the memory storage for model parameters and additional runtime storage for intermediate outputs should be within the memory constraints of the embedded device.
This is achieved in two ways: (a) reducing the precision of the parameters and intermediate outputs and (b) pruning the parameters by increasing sparsity.
\end{itemize}

\subsection{Optimization for Efficiency}

\red{Here, in the view of the above efficiency requirements, we compare the three baseline algorithms.
We then select an efficient design scheme for NNs that satisfies all our requirements.}
\virat{why is the choice of design scheme in motivation?}

\noindent\textbf{Memory Efficiency.} Off the shelf models are designed to specifically reduce the memory footprint.
For instance, the memory footprint of Squeezenet and MobileNet is 5MB and 14Mb compared to 250Mb of Alexnet and >500Mb of VGG architectures.
Additionally, lowering the model precision from 64 or 32 bit floating point to binary precision results in a direct reduction of 64x or 32x in the overall memory footprint of the model.
However, in case of model compression the model parameters which are pruned are simply replaced by a value of "0".
Hence, storing even the "0" parameter takes up memory and does not necessarily decrease the overall memory footprint unless the hardware is optimized to skip the storage of all the zero values in the memory.
This requires additional logic to check for zero valued parameters in a dictionary.

\begin{figure*}[ht!]
\begin{center}% note that \centering uses less vspace...
\resizebox{2\columnwidth}{!}{%
\begin{tabular}{lllll}


\begin{tikzpicture}
% let both axes use the same layers
\pgfplotsset{set layers}
%
\begin{axis}[
scale only axis,
line width=2.0pt,
mark size=2.0pt,
xmin=0,xmax=8,
ylabel={Generalization Error},
xlabel style={font=\LARGE},
ylabel style = {font=\LARGE},
axis y line*=left,
xlabel={Compression Rate},
xticklabel style = {font=\large},
yticklabel style = {font=\large},
xtick={0,1,2,3,4,5,6,7,8},
xticklabels={1x, 1.33x, 1.82x, 2.56x, 3.67x, 5.31x, 7.74x, 11.28x, 16.28x},
legend style={at={(0.98,0.9)},anchor=north east, font=\small}
]
\addplot[
    color=black,
    solid,
    mark=*,
    mark options={solid},
    smooth
    ]
    coordinates {
    (0,7.99)(1,7.84)(2,7.4)(3,6.85)(4,5.44)(5,4.45)(6,3.49)(7,2.58)(8,2.43)
      };
\addlegendimage{color=black,solid,mark=*, mark options={solid}}
\addlegendentry{Generalization Error}
\end{axis}

\begin{axis}[
scale only axis,
line width=2.0pt,
mark size=2.0pt,
xmin=0,xmax=8,
ylabel near ticks, yticklabel pos=right,
ylabel={Inference Accuracy},
ylabel style = {rotate=180, font=\LARGE},
yticklabel style = {font=\large},
axis x line=none
]
\addplot[
    color=black,
    dashed,
    mark=*,
    mark options={solid},
    smooth
    ]
    coordinates {
    (0,54.32)(1,54.33)(2,54.11)(3,53.68)(4,52.80)(5,52.31)(6,51.88)(7,51.36)(8,51.31)
        };
\addlegendimage{color=black,dashed,mark=*,mark options={solid}}
\addlegendentry{Inference Accuracy}
\end{axis}
\end{tikzpicture} &

%
\begin{tikzpicture}
% let both axes use the same layers
\pgfplotsset{set layers}
%
\begin{axis}[
scale only axis,
line width=2.0pt,
mark size=2.0pt,
xmin=0,xmax=8,
ylabel={Generalization Error},
axis y line*=left,
xlabel={Compression Rate},
xlabel style={font=\LARGE},
ylabel style = {font=\LARGE},
xticklabel style = {font=\large},
yticklabel style = {font=\large},
xtick={0,1,2,3,4,5,6,7,8},
xticklabels={1.00x, 1.21x, 1.51x, 1.92x, 2.55x, 3.06x, 4.64x, 6.07x, 7.57x},
legend style={at={(0.02,0.1)},anchor=south west, font=\small}
]
\addplot[
    color=black,
    solid,
    mark=*,
    mark options={solid},
    smooth
    ]
    coordinates {
    (0,38.5)(1,38.36)(2,38.36)(3,38.23)(4,38.95)(5,39.56)(6,37.01)(7,26.72)(8,15.48)
      };
\addlegendimage{color=black,solid,mark=*, mark options={solid}}
\addlegendentry{Generalization Error}
\end{axis}

\begin{axis}[
scale only axis,
line width=2.0pt,
mark size=2.0pt,
xmin=0,xmax=8,
ylabel near ticks, yticklabel pos=right,
ylabel={Inference Accuracy},
ylabel style = {rotate=180, font=\LARGE},
yticklabel style = {font=\large},
axis x line=none,
legend style={at={(0.02,0)},anchor=south west, font=\small}
]
\addplot[
    color=black,
    dashed,
    mark=*,
    mark options={solid},
    smooth
    ]
    coordinates {
    (0,83.43)(1,83.42)(2,83.03)(3,81.48)(4,78.72)(5,80.83)(6,69.59)(7,63.81)(8,59.10)
        };
\addlegendimage{color=black,dashed,mark=*,mark options={solid}}
\addlegendentry{Inference Accuracy}
\end{axis}
\end{tikzpicture} &





%
\begin{tikzpicture}
% let both axes use the same layers
\pgfplotsset{set layers}
%
\begin{axis}[
scale only axis,
line width=2.0pt,
mark size=2.0pt,
xmin=0,xmax=8,
ylabel={Generalization Error},
axis y line*=left,
xlabel={Compression Rate},
xlabel style={font=\LARGE},
ylabel style = {font=\LARGE},
xticklabel style = {font=\large},
yticklabel style = {font=\large},
xtick={0,1,2,3,4,5,6,7,8},
xticklabels={1.00x, 1.23x, 1.57x, 2.09x, 2.91x, 4.21x, 6.22x, 9.16x, 12.93x},
legend style={at={(0.02,0.1)},anchor=south west, font=\small}
]
\addplot[
    color=black,
    solid,
    mark=*,
    mark options={solid},
    smooth
    ]
    coordinates {
    (0,32.47)(1,32.72)(2,33)(3,33.56)(4,34.3)(5,32.03)(6,17.48)(7,7.76)(8,3.01)
      };
\addlegendimage{color=black,solid,mark=*, mark options={solid}}
\addlegendentry{Generalization Error}
\end{axis}

\begin{axis}[
scale only axis,
line width=2.0pt,
mark size=2.0pt,
xmin=0,xmax=8,
ylabel near ticks, yticklabel pos=right,
ylabel={Inference Accuracy},
ylabel style = {rotate=180,font=\LARGE},
yticklabel style = {font=\large},
axis x line=none,
legend style={at={(0.02,0)},anchor=south west, font=\small}
]
\addplot[
    color=black,
    dashed,
    mark=*,
    mark options={solid},
    smooth
    ]
    coordinates {
    (0,83.30)(1,83.32)(2,83.16)(3,81.11)(4,75.86)(5,67.97)(6,59.31)(7,54.60)(8,52.63)
        };
\addlegendimage{color=black,dashed,mark=*,mark options={solid}}
\addlegendentry{Inference Accuracy}
\end{axis}
\end{tikzpicture}


\end{tabular}
}
\caption{\underline{Pruning reduces Inference Accuracy at the cost of Prediction Accuracy.} On pruning the models trained on FashionMNIST(left), Purchase100(Center) and Location(Right) dataset, we find that the inference accuracy decreases due to the decrease in the generalisaton error and lower prediction accuracy.}
\label{fig:prune}
\end{center}
\end{figure*}

\begin{figure*}[ht!]
\begin{center}% note that \centering uses less vspace...
\resizebox{2\columnwidth}{!}{%
\begin{tabular}{lllll}


\begin{tikzpicture}
% let both axes use the same layers
\pgfplotsset{set layers}
%
\begin{axis}[
scale only axis,
line width=2.0pt,
mark size=2.0pt,
xmin=0,xmax=8,
ylabel={Generalization Error},
axis y line*=left,
xlabel={Compression Rate},
xtick={0,1,2,3,4,5,6,7,8},
xticklabels={1x, 1.33x, 1.82x, 2.56x, 3.67x, 5.31x, 7.74x, 11.28x, 16.28x}
]
\addplot[
    color=black,
    solid,
    mark=*,
    mark options={solid},
    smooth
    ]
    coordinates {
    (0,7.99)(1,7.84)(2,7.4)(3,6.85)(4,5.44)(5,4.45)(6,3.49)(7,2.58)(8,2.43)
      };
\end{axis}

\begin{axis}[
scale only axis,
line width=2.0pt,
mark size=2.0pt,
xmin=0,xmax=8,
ylabel near ticks, yticklabel pos=right,
ylabel={Inference Accuracy},
ylabel style = {rotate=180},
axis x line=none
]
\addplot[
    color=black,
    dashed,
    mark=*,
    mark options={solid},
    smooth
    ]
    coordinates {
    (0,54.32)(1,54.33)(2,54.11)(3,53.68)(4,52.80)(5,52.31)(6,51.88)(7,51.36)(8,51.31)
        };
\end{axis}
\end{tikzpicture} &

%
\begin{tikzpicture}
% let both axes use the same layers
\pgfplotsset{set layers}
%
\begin{axis}[
scale only axis,
line width=2.0pt,
mark size=2.0pt,
xmin=0,xmax=8,
ylabel={Generalization Error},
axis y line*=left,
xlabel={Compression Rate},
xtick={0,1,2,3,4,5,6,7,8},
xticklabels={1.00x, 1.21x, 1.51x, 1.92x, 2.55x, 3.06x, 4.64x, 6.07x, 7.57x}
]
\addplot[
    color=black,
    solid,
    mark=*,
    mark options={solid},
    smooth
    ]
    coordinates {
    (0,38.5)(1,38.36)(2,38.36)(3,38.23)(4,38.95)(5,39.56)(6,37.01)(7,26.72)(8,15.48)
      };
\end{axis}

\begin{axis}[
scale only axis,
line width=2.0pt,
mark size=2.0pt,
xmin=0,xmax=8,
ylabel near ticks, yticklabel pos=right,
ylabel={Inference Accuracy},
ylabel style = {rotate=180},
axis x line=none
]
\addplot[
    color=black,
    dashed,
    mark=*,
    mark options={solid},
    smooth
    ]
    coordinates {
    (0,83.43)(1,83.42)(2,83.03)(3,81.48)(4,78.72)(5,80.83)(6,69.59)(7,63.81)(8,59.10)
        };
\end{axis}
\end{tikzpicture} &





%
\begin{tikzpicture}
% let both axes use the same layers
\pgfplotsset{set layers}
%
\begin{axis}[
scale only axis,
line width=2.0pt,
mark size=2.0pt,
xmin=0,xmax=8,
ylabel={Generalization Error},
axis y line*=left,
xlabel={Compression Rate},
xtick={0,1,2,3,4,5,6,7,8},
xticklabels={1.00x, 1.23x, 1.57x, 2.09x, 2.91x, 4.21x, 6.22x, 9.16x, 12.93x}
]
\addplot[
    color=black,
    solid,
    mark=*,
    mark options={solid},
    smooth
    ]
    coordinates {
    (0,32.47)(1,32.72)(2,33)(3,33.56)(4,34.3)(5,32.03)(6,17.48)(7,7.76)(8,3.01)
      };
\end{axis}

\begin{axis}[
scale only axis,
line width=2.0pt,
mark size=2.0pt,
xmin=0,xmax=8,
ylabel near ticks, yticklabel pos=right,
ylabel={Inference Accuracy},
ylabel style = {rotate=180},
axis x line=none
]
\addplot[
    color=black,
    dashed,
    mark=*,
    mark options={solid},
    smooth
    ]
    coordinates {
    (0,83.30)(1,83.32)(2,83.16)(3,81.11)(4,75.86)(5,67.97)(6,59.31)(7,54.60)(8,52.63)
        };
\end{axis}
\end{tikzpicture}


\end{tabular}
}
\caption{FashionMNIST(left) Purchase100(Center) Location(Right). Dashed is Inference accuracy, solid is generalisaton error}
\label{fig:loss}
\end{center}
\end{figure*}

\begin{figure*}[ht!]
\begin{center}% note that \centering uses less vspace...
\resizebox{2\columnwidth}{!}{%
\begin{tabular}{lllll}


\begin{tikzpicture}
% let both axes use the same layers
\pgfplotsset{set layers}
%
\begin{axis}[
title={(a) FashionMNIST},
title style={at={(0.5,0)},anchor=north,yshift=-40, font=\huge},
scale only axis,
line width=2.0pt,
mark size=2.0pt,
xmin=0,xmax=4,
ylabel={Generalization Error},
axis y line*=left,
xlabel={Precision},
xtick={0,1,2,3,4},
xlabel style={font=\LARGE},
ylabel style = {font=\LARGE},
xticklabel style = {font=\large},
yticklabel style = {font=\large},
xticklabels={32b, 8b, 4b, 3b, 2b},
legend style={at={(0.02,0.1)},anchor=south west, font=\small}
]
\addplot[
    color=black,
    solid,
    mark=*,
    mark options={solid},
    smooth
    ]
    coordinates {
    (0,11.42)(1,11.55)(2,11.03)(3,8.28)(4,5.05)
      };
      \addlegendimage{color=black,solid,mark=*, mark options={solid}}
      \addlegendentry{Generalization Error}
\end{axis}

\begin{axis}[
scale only axis,
line width=2.0pt,
mark size=2.0pt,
xmin=0,xmax=4,
ylabel near ticks, yticklabel pos=right,
ylabel={Inference Accuracy},
ylabel style = {rotate=180, font=\LARGE},
yticklabel style = {font=\large},
axis x line=none,
legend style={at={(0.02,0)},anchor=south west, font=\small}
]
\addplot[
    color=black,
    dashed,
    mark=*,
    mark options={solid},
    smooth
    ]
    coordinates {
    (0,56.57)(1,56.54)(2,55.67)(3,54.24)(4,52.64)
        };
        \addlegendimage{color=black,dashed,mark=*,mark options={solid}}
        \addlegendentry{Inference Accuracy}
\end{axis}
\end{tikzpicture} &

%
\begin{tikzpicture}
% let both axes use the same layers
\pgfplotsset{set layers}
%
\begin{axis}[
title={(b) Purchase},
title style={at={(0.5,0)},anchor=north,yshift=-40, font=\huge},
scale only axis,
line width=2.0pt,
mark size=2.0pt,
xmin=0,xmax=4,
ylabel={Generalization Error},
axis y line*=left,
xlabel={Precision},
xtick={0,1,2,3,4},
xlabel style={font=\LARGE},
ylabel style = {font=\LARGE},
xticklabel style = {font=\large},
yticklabel style = {font=\large},
xticklabels={32b, 8b, 4b, 3b, 2b},
legend style={at={(0.02,0.1)},anchor=south west, font=\small}
]
\addplot[
    color=black,
    solid,
    mark=*,
    mark options={solid},
    smooth
    ]
    coordinates {
    (0,20.26)(1,20.31)(2,18.57)(3,14.43)(4,6.45)
      };
      \addlegendimage{color=black,solid,mark=*, mark options={solid}}
      \addlegendentry{Generalization Error}
\end{axis}

\begin{axis}[
scale only axis,
line width=2.0pt,
mark size=2.0pt,
xmin=0,xmax=4,
ylabel near ticks, yticklabel pos=right,
ylabel={Inference Accuracy},
ylabel style = {rotate=180,font=\LARGE},
yticklabel style = {font=\large},
axis x line=none,
legend style={at={(0.02,0)},anchor=south west, font=\small}
]
\addplot[
    color=black,
    dashed,
    mark=*,
    mark options={solid},
    smooth
    ]
    coordinates {
    %(0,62.08)(1,62.05)(2,59.86)(3,57.27)(4,53.38)
    (0,84.45)(1,84.22)(2,76.27)(3,71.32)(4,62.81) %updated results
        };
        \addlegendimage{color=black,dashed,mark=*,mark options={solid}}
        \addlegendentry{Inference Accuracy}
\end{axis}
\end{tikzpicture} &





%
\begin{tikzpicture}
% let both axes use the same layers
\pgfplotsset{set layers}
%
\begin{axis}[
title={(c) Location},
title style={at={(0.5,0)},anchor=north,yshift=-40, font=\huge},
scale only axis,
line width=2.0pt,
mark size=2.0pt,
xmin=0,xmax=4,
ylabel={Generalization Error},
axis y line*=left,
xlabel={Precision},
xtick={0,1,2,3,4},
xlabel style={font=\LARGE},
ylabel style = {font=\LARGE},
xticklabel style = {font=\large},
yticklabel style = {font=\large},
xticklabels={32b, 8b, 4b, 3b, 2b},
legend style={at={(0.02,0.1)},anchor=south west, font=\small}
]
\addplot[
    color=black,
    solid,
    mark=*,
    mark options={solid},
    smooth
    ]
    coordinates {
    (0,37.48)(1,37.57)(2,38.28)(3,37.70)(4,19.93)
      };
      \addlegendimage{color=black,solid,mark=*, mark options={solid}}
      \addlegendentry{Generalization Error}
\end{axis}

\begin{axis}[
scale only axis,
line width=2.0pt,
mark size=2.0pt,
xmin=0,xmax=4,
ylabel near ticks, yticklabel pos=right,
ylabel={Inference Accuracy},
ylabel style = {rotate=180,font=\LARGE},
yticklabel style = {font=\large},
axis x line=none,
legend style={at={(0.02,0)},anchor=south west, font=\small}
]
\addplot[
    color=black,
    dashed,
    mark=*,
    mark options={solid},
    smooth
    ]
    coordinates {
    (0,81.82)(1,81.77)(2,76.07)(3,71.61)(4,61.95)
        };
        \addlegendimage{color=black,dashed,mark=*,mark options={solid}}
        \addlegendentry{Inference Accuracy}
\end{axis}
\end{tikzpicture}


\end{tabular}
}
\caption{Pruning followed by quantization (i.e., weight sharing) restores the predictive accuracy while limiting information leakage (i.e., generalization error).}
%\caption{\underline{Pruning followed by Weight Sharing (Quantization).} While the retraining after pruning is necessary to restore the predictive accuracy, clustering the weights to reduce the precision lowers the inference accuracy risk while reducing the generalization error for FashionMNIST(left), Purchase100(Center) and Location(Right) dataset.}
\label{fig:wtsharing}
\end{center}
\end{figure*}


\noindent\textbf{Computation Efficiency.} Design of efficient off-the-shelf architectures replaces the complex matrix-vector multiplications by multiple matrix-vector multiplications with smaller dimensions.
This reduces the overall number of parameters but it has been shown empirically\footnote{https://github.com/albanie/convnet-burden} that this does not necessarily reduce the number of multiply accumulate operations or FLOPS~\cite{article}.
In case of parameter pruning, achieving efficiency requires additional hardware optimization. Particularly, instead of actually computing the the multiplications with "0"pruned values, the hardware optimization enable the user to skip the computation and replace the output by a "0" directly.
For quantized models with binarized parameters and activations the MAC operations can be replaced by binary operations such as XNOR and the maxpool operations can be replaced by OR operation, while the activations can be replaced by checking the parity bit (denotes the sign) operation and hence reducing the FLOPS drastically[XONN].
This results in high computational efficiency and hence, faster inference.


\noindent\textbf{Energy Efficiency.} Energy efficiency has been shown to not vary much in case of reduction with number of parameters and data type, number of memory accesses play vital role[CVPR 2017 yang et al][].
Specifically, for the case of off-the-shelf architectures the while the computation efficiency has been shown to improve, the energy efficiency has been shown to be close to large scale state of the art models like AlexNet[suqeezenet][].
Alternatively, for the case of model compression, energy efficiency can be achieved by additionally providing hardware optimization and shows small improvement in the energy consumption[].
For quantization, however, the energy efficiency has been shown to be high [][] where the memory access can be drastically reduced by increasing the throughput of data fetched from the memory.
Specifically, lowering the precision from 32 bit floating point to binary results in lowering the memory accesses and 32x improvement in energy consumption[][][].
While some improvement is seen natively for quantized models (from replacing MACs with XNOR), higher benefits can be achieved via additional hardware optimization[].
The benchmarking of energy consumption for different optimization and architectures is well explored in the literature and out of scope of this work. We refer the authors to [][][] for more details.

In summary, compared to different optimization techniques for NNs, the quantized architectures show significant benefit for different efficiency requirements over the other alternatives.


\subsection{Evaluating Privacy Leakage}

In this section, we evaluate the information leakage through membership inference attacks for the three baseline algorithms considered.
This is the main contribution of our work where we evaluate the privacy leakage for different optimization and design algorithms for NNs.

\subsubsection{Model Compression}

We evaluate the privacy leakage on compressing a model by pruning the connections in the model.
Here, pruning is achieved by replacing some of the parameters with "0" value.
As described in the original paper~\cite{}, pruning is followed by retraining the model to restore the model's original accuracy with the pruned connections.

We evaluate and validate the impact on membership privacy on compressing the model trained on three datasets: FashionMNIST, Location and Purchase100.
On pruning the model, the model's test accuracy decreases but also lowers the membership inference accuracy (Figure~\ref{fig:prune}).
This is expected as the parameters are responsible for memorizing the training data information and pruning the parameters lowers the adversary's attack success~\cite{rezawhite}.

However, interestingly, on retraining the pruned model, we observe that the membership inference accuracy is much higher than the original unpruned baseline model (Figure~\ref{fig:retrain}).
This indicates that model compression in turn increases the overall privacy leakage.
This can be attributed to the lower number of parameters forced to learn the same amount of information stored previously in the unpruned model with larger number of parameters.
In other words, the same amount of information is now captured by less number of parameters resulting in higher memorization of information per parameter.
To analyze the information stored per parameter, we first compute the model capacity as the mutual information of a trained network between the true label $Y$ and the predicted label $Y_{\theta}$ for a random input $X$ as derived in~\cite{45932,cap}.
Here, model's information $I(Y;Y_{\theta}|X) = $
\begin{equation}
N_{train}\left(1 - (r_{train}log_2(\frac{1}{r_{train}}) + (1-r_{train})log_2(\frac{1}{1-r_{train}}))\right)
\end{equation}
where $r_{train}$ is the classification train accuracy for all the $N_{train}$ samples in the training data.
For $r_{train} = 1$, the model completely memorizes all random samples as the information stored equals the number of samples $N_{train}$, while for $r_{train}=0.5$, the training accuracy 0.
We divide the above equation by the model's total number of parameters $N_{param}$ to get the per parameter information stored as
\begin{equation}
I_p(Y;Y_{\theta}|X) = \frac{I(Y;Y_{\theta}|X)}{N_{param}}
\end{equation}
As the model is compressed (pruned), the number of parameters $N_{param}$ decreases which results in increase in $I_p(Y;Y_{\theta}|X)$. However, on aggressive pruning, the train accuracy $r_{train}$ also decreases resulting in a decrease in the information per parameter, which is empirically indicated by a decrease in membership inference accuracy at the end in Figure~\ref{fig:retrain}.


\textbf{Mitigating the Privacy Risks in Pruned Models.} We describe on potential approach to mitigate the privacy risk of the compressed models without requiring to modify the model's training.
The post-hoc apporach utilizes the weight sharing for the compressed model. This is however, accompanied by a decrease in the model's prediction accuracy indicating a privacy-utility trade-off.
As seen in Figure~\ref{fig:wtsharing}, reducing the precision from 32 bits to 2 bits results in a decrease in inference accuracy from 56.57\% to 52.64\% for FashionMNIST, 62.08\% to 53.38\% for Location and 81.82\% to 61.95\% for Purchase100 dataset.
This decrease in inference attack accuracy is closely followed by a decrease in generalization error which is indicative of decrease in prediction accuracy of the model.
We evaluate the effectiveness of pruning followed by quantization which has been shown to have significant impact on reducing the model complexity through compression more significantly than either pruning or quantization alone.
For the experiments, we use the compressed model indicating highest privacy leakage to evaluate the effectiveness of weight sharing on the worst case condition.
This pipelined approach of pruning followed by retraining followed by weight sharing, not only maintains the algorithm's objective for efficiency but is used as a post-hoc approach to reduces the overall inference risk~\cite{DBLP:journals/corr/HanMD15,DBLP:journals/corr/HanPNMTECTD16}.






\subsubsection{Off-the-Shelf Efficient Architectures}


\begin{table}[!htb]
\begin{center}
\renewcommand\arraystretch{1.5}
\fontsize{6.7pt}{6.7pt}\selectfont
\begin{tabular}{|c|c|c|c|c|}
\hline
\textbf{Architecture} & \textbf{Memory} & \textbf{Train}  & \textbf{Test}  & \textbf{Inference}   \\
 & \textbf{Footprint} & \textbf{Accuracy} & \textbf{Accuracy} & \textbf{Accuracy}  \\
\hline
SqueezeNet & 5 MB & 88.21\% & 81.92\% & \cellcolor{green!25}53.07\% \\
MobileNetV2 & 14 MB & 97.50\% & 87.24\% & \cellcolor{green!25}55.57\% \\
\hline
AlexNet & 240 MB & 97.86\% & 80.34\% & \cellcolor{red!25}60.40\% \\
VGG11 & 507 MB & 99.13\% & 86.43\% & \cellcolor{red!25}58.04\% \\
VGG16 & 528 MB & 99.58\% & 88.95\% & \cellcolor{red!25}58.70\%  \\
VGG19 & 549 MB & 99.09\% & 88.18\% & \cellcolor{red!25}57.85\% \\
\hline
\end{tabular}
\end{center}
\caption{Model complexity influences the membership inference leakage. Model specifically designed for efficiency leak less information.}
\label{stdarch}
\end{table}

In this section, we evaluate two popular state of the art architectures, SqueezeNet and MobileNet, trained on CIFAR10 dataset used for low powered systems.
As seen in Table~\ref{stdarch}, the SqueezeNet and MobileNet models shows lower inference accuracy of 53.07\% and 55.57\% compared to larger models which have higher privacy leakage.

\begin{figure}[hb!]
\resizebox{0.6\columnwidth}{!}{%
\begin{tikzpicture}
\begin{axis}[
legend style={font=\small},
legend pos =  north east,
line width=1.0pt,
mark size=1.0pt,
ymin=50,
legend entries={MobileNetV2, SqueezeNet},
ylabel={Inference Accuracy},
xlabel={Temperature Parameter},
xlabel style={font=\large},
ylabel style={font=\large},
% extra x ticks={1,10,...,400},
% extra y ticks={0,0.5,...,10},
% extra y tick labels={},
% extra x tick labels={},
% extra x tick style={grid=major},
% extra y tick style={grid=major},
grid=major
]
\addplot[
    color=black,
    solid,
    mark=*,
    mark options={solid},
    smooth
    ]
    coordinates {
    (1,55.57)(5,54.25)(10,53.21)(15,52.85)(20,52.62)
      };
\addplot[
      color=black,
      dashed,
      mark=*,
      mark options={solid},
      smooth
    ]
    coordinates {
    (1,53.25)(5,51.37)(10,51.08)(15,50.99)(20,50.93)
      };
\end{axis}
\end{tikzpicture}
}
\caption{The privacy leakage of off-the-shelf models is reduced by increasing the softmax temperature.}
\label{softmax}
\end{figure}


Further, the inference accuracy of SqueezeNet and MobileNet can be further reduced close to random guess by increasing the temperature parameter of the softmax function applied to the output.
Increasing the temperature parameter reduces the granularity of the model's output and is given by $F_i(x) = \frac{e^{\frac{z_i(x)}{T}}}{\sum_{j}e^{\frac{z_j(x)}{T}}}$ where $z(x)$ computes output of the model before the softmax layer.
For the case of SqueezeNet, we are able to reduce the inference accuracy to 50.93\% from 53.07\% while for MobileNet we can reduce the inference accuracy to 52.62\% from 55.57\% as seen in Figure~\ref{softmax}.
This reduction in inference accuracy is without any cost of the prediction test accuracy of the model.




\subsubsection{Quantization}

In this section, we evaluate the technique of reducing the precision of both model's parameters and intermediate activations.
Further, we consider the extreme case of binarizing the parameters and activations allowing to evaluate on the most optimized case.
We evaluate on FashionMNIST dataset for two architectures with convolutional and fully connected layers.

\begin{table}[!htb]
\begin{center}
\renewcommand\arraystretch{1.5}
\fontsize{6.7pt}{6.7pt}\selectfont
\begin{tabular}{|c|c|c|c|c|}
\hline
\multicolumn{5}{|c|}{\textbf{FashionMNIST}}\\
\hline
\textbf{Architecture} & \textbf{Memory} & \textbf{Train}  & \textbf{Test}  & \textbf{Inference}  \\
 & \textbf{Accuracy} &  \textbf{Footprint} & \textbf{Accuracy} & \textbf{Accuracy}  \\
\hline
\multicolumn{5}{|c|}{Architecture 1}\\
Full & 38.39 MB & 100\% & 92.35\% & \cellcolor{red!25}57.46\%\\
BinaryNet & 1.62 MB & 88.68\% & 86.9\% & \cellcolor{green!25}55.45\%\\
XNOR-Net & 1.62 MB & 87.19\% & 85.68\% & \cellcolor{green!25}51.05\%\\ %1,626,824 parameters
\hline
\multicolumn{5}{|c|}{Architecture 2}\\
Full & 29.83 MB & 99.34\% & 89.88\% & \cellcolor{red!25}54.86\% \\
BinaryNet & 0.93 MB & 97.61\% & 89.60\% & \cellcolor{green!25}54.30\%\\
XNOR-Net & 0.93 MB & 92.67\% & 86.68\% & \cellcolor{green!25}51.74\%\\ %937,000parameters
\hline
\end{tabular}
\end{center}
\caption{Reducing the model precision lowers the inference attack accuracy but at the cost of test accuracy.}
\label{fmnist_quantize}
\end{table}

In both the architectures, we see that computation on  binarized parameters and activations reduces the inference risk by a small value.
However, on replacing the MAC operations with XNOR operations, we observe that the inference risk decreases close to random guess, however, at the cost of prediction test accuracy.

In summary, we observe that quantization, specifically binarization of parameters and activation along with XNOR computation, provides strong resistance against inference attacks compared to model compression and off-the-shelf architectures.

\subsection{Summary of Comparison}

We summarize the properties satisfied by each of the technique in terms of privacy, computation, memory and energy efficiency in Table~\ref{tbl:comparison}.
Here, we mark the attributes which are satisfied with $\cmark$, requires additional hardware optimization as $\smark$ and does not satisfy the property with a $\xmark$.

\begin{table}[!htb]
\begin{center}
\renewcommand\arraystretch{1.5}
\fontsize{6.7pt}{6.7pt}\selectfont
\begin{tabular}{|l||l|l|l|}
\hline
Requirements & Compression & Quantization & Off-the-shelf  \\
\hline
Computation Efficiency & $\smark$  & $\cmark$   & $\xmark$ \\
\hline
Memory Efficiency &  $\smark$ & $\cmark$   & $\cmark$ \\
\hline
Energy Efficiency &  $\smark$   & $\cmark$   & $\xmark$ \\
\hline
Privacy &  $\xmark$   & $\cmark$   & $\smark$ \\
\hline
\end{tabular}
\end{center}
\caption{Comparison of different optimizations for NNs. $\smark$: additional hardware optimization; $\xmark$: requirement not satisfied; $\cmark$: requirement satisfied.}
\label{tbl:comparison}
\end{table}

For our requirement, quantization of NNs is an attractive design choice which not only satisfies the computation, memory and energy efficiency but also provides high resistance against inference attacks.
Specifically, in this work we only consider the aggressive quantization of binarizing the parameters and activations to \{-1,+1\} values while additionally replacing the MAC operations with cheap and efficient binary arithmetic (XNOR operations).
Hence, we choose this particular design for NNs to provide a good three dimensional trade-off between privacy-efficiency-accuracy.

\section{\method: Design Overview}\label{design}

Based on the efficiency and privacy analysis described in the previous section, we describe the detailed \method\hspace{0.02in} framework for designing efficient, private and accurate NNs in this section.
In Phase I, the objective is to enhance the model's efficiency and privacy, however, at the cost of accuracy.
In Phase II, we optimize for accuracy and train the resultant model from Phase I.


\subsection{Phase I}

Computation Efficiency:
Instead of using multiplication and addition circuits, we perform XNOR operations on the inputs followed by a bitcount operation. This reduces the overall number of non-XOR gates used to compute the operation. The equation can be represented as follows.

\begin{align}
\mathbf{x} \cdot \mathbf{w} =
N - 2\times\operatorname{bitcount}(\operatorname{xnor}(\mathbf{x}, \mathbf{w}))
\end{align}

As our main contribution, we show that neural network algorithms can be heavily optimized to execute efficiently using garbled circuits. We observe that the efficiency of evaluating an inference circuit depends on two key factors: the model parameters and the network structure.  With this observation,  selects optimal parameter size and network structure to guarantee acceptable {\em performance}. Last but not the least, to ensure high {\em accuracy} of the model,  uses an architecture search approach to find the best model with high accuracy and efficiency on garbled circuits.

One approach of improving efficiency for Neural Networks is through quantizing the parameter values and hence, reducing the precision of parameters number of bits required to represent the values in hardware \cite{Hubara:2017:QNN:3122009.3242044}.

A specific case of quantized Neural networks is using binary parameter and activation values\{-1,+1\} \cite{NIPS2016_6573}\cite{NIPS2015_5647}
For such binarized Neural Networks, the computation overhead due to matrix multiplication can be replaced by cheaper XNOR computation \cite{rastegari2016xnornet}\cite{DBLP:journals/corr/ZhouNZWWZ16}.
BNNs result in lower accuracy compared to full precision counterparts and several research papers and explored improving the accuracy-efficiency trade-off \cite{AAAI1714619}.
Ternary weighted Networks provide better accuracy compared to BNNs at the cost of higher precision with weights \{-W,0,+W\} where the threshold $W$ can be learned for higher performance \cite{DBLP:journals/corr/ZhuHMD16}\cite{Li2016TernaryWN}

\begin{algorithm}
\begin{algorithmic}
    \FOR{$k=1$ to $L$}
        \STATE $W_k^b \leftarrow {\rm Binarize}(W_k)$
        \STATE $a_k \leftarrow a_{k-1}^b W_k^b$
        \IF{$k < L$}
            \STATE $a_k^b \leftarrow {\rm Binarize}(a_k)$
        \ENDIF
    \ENDFOR

\end{algorithmic}
\caption{
Inference Stage of Binary Neural Network; Binarize() function is deterministic thresholding scheme; $W_k^b$ are the binarized weights($W_k$) and $a_k$ is the activation of the $k^{th}$ layer
}
\label{alg:train}
\end{algorithm}

Memory Efficiency: Quantization of parameters reduces the memory overhead by 32x to 64x.

Energy Efficiency: These optimization result in reduceing the
Two such Neural Network models are SqueezeNet \cite{DBLP:journals/corr/IandolaMAHDK16} and MobileNetV2 \cite{conf/cvpr/SandlerHZZC18} which are specifically designed to have less number of parameters and memory footprint. (does not ensure energy efficiency)

Dense sparse dense training \cite{DBLP:journals/corr/HanPNMTECTD16}
Model compression pipeline cobining pruning, quantization and huffman coding \cite{DBLP:journals/corr/HanMD15}
Huffman Coding is used for representation while storing in the hardware and in our experiments, we use pruning followed by quantization to achieve model compression.


In this paper, we provide directions to design neural networks for secure inference and low private computation overhead. We show that our optimized neural network architectures execute faster than Gazelle, the most efficient existing solution on privacy-preserving deep learning inference.

Specific optimization for XNORNet to avoid a loss in accuracy....


Difficult to train binarized model~\cite{AAAI1714619}




\subsection{Phase II}


Describe model distillation
\pgfdeclarelayer{background}
\pgfdeclarelayer{foreground}
\pgfsetlayers{background,main,foreground}


\tikzstyle{input} = [rectangle, rounded corners, minimum width=0.5cm, minimum height=3cm,text centered, draw=black, fill=gray!15]
\tikzstyle{layer11} = [rectangle, rounded corners, minimum width=0.25cm, minimum height=4.5cm,text centered, draw=black, fill=gray!15]
\tikzstyle{layer12} = [rectangle, rounded corners, minimum width=0.25cm, minimum height=3.5cm,text centered, draw=black, fill=gray!15]
\tikzstyle{layer13} = [rectangle, rounded corners, minimum width=0.25cm, minimum height=2.5cm,text centered, draw=black, fill=gray!15]
\tikzstyle{layer14} = [rectangle, rounded corners, minimum width=0.25cm, minimum height=1.5cm,text centered, draw=black, fill=gray!15]



\tikzstyle{input} = [rectangle, rounded corners, minimum width=0.5cm, minimum height=3cm,text centered, draw=black, fill=gray!15]
\tikzstyle{layer1} = [rectangle, rounded corners, minimum width=0.25cm, minimum height=3cm,text centered, draw=black, fill=gray!15]
\tikzstyle{layer2} = [rectangle, rounded corners, minimum width=0.25cm, minimum height=2cm,text centered, draw=black, fill=gray!15]
\tikzstyle{layer3} = [rectangle, rounded corners, minimum width=0.25cm, minimum height=1.5cm,text centered, draw=black, fill=gray!15]
\tikzstyle{layer4} = [rectangle, rounded corners, minimum width=0.25cm, minimum height=1cm,text centered, draw=black, fill=gray!15]
\tikzstyle{write} = [rectangle, rounded corners, minimum width=2.5cm, minimum height=1.5cm,text centered, draw=black, fill=gray!15]
\tikzstyle{neuron}=[circle,draw=black, fill=gray!15,minimum size=8pt,inner sep=0pt]
\tikzstyle{hidden neuron}=[neuron, draw=black, fill=gray!15]
\tikzstyle{output neuron}=[neuron, draw=black, fill=gray!15]

\begin{figure}[!htb]
\centering
\resizebox{\columnwidth}{!}{%
\begin{tikzpicture}[node distance=2cm, line width=1pt,every node/.style={align=center}]




\node (teach1) [layer11]  at (-9.75,0) {};
\node (teach2) [layer12]  at (-9.25,0) {};
\node (teach3) [layer13]  at (-8.75,0) {};
\node (teach4) [layer14]  at (-8.25,0) {};


\path[yshift=1.5cm, xshift=-0.5cm] node[hidden neuron] (H11) at (-7,-0.5 cm) {};
\path[yshift=1.5cm, xshift=-0.5cm]node[hidden neuron] (H12) at (-7,-1 cm) {};
\path[yshift=1.5cm, xshift=-0.5cm] node[hidden neuron] (H13) at (-7,-1.5 cm) {};
\path[yshift=1.5cm, xshift=-0.5cm]node[hidden neuron] (H14) at (-7,-2 cm) {};
\path[yshift=1.5cm, xshift=-0.5cm] node[hidden neuron] (H15) at (-7,-2.5 cm) {};

\path[yshift=1.5cm, xshift=-0.5cm] node[output neuron] (O11) at (-6.5,-1 cm) {};
\path[yshift=1.5cm, xshift=-0.5cm] node[output neuron] (O12) at (-6.5,-1.5 cm) {};
\path[yshift=1.5cm, xshift=-0.5cm] node[output neuron] (O13) at (-6.5,-2 cm) {};

\node (out1) [output neuron, right of=H13, xshift=-1cm] {};
\begin{scope}[on background layer]
    \node (teacher) [fit=(teach1) (teach2) (teach3) (teach4) (H11) (H12) (H13) (H14) (H15) (O11) (O12) (O13) (out1), fill= gray!20, rounded corners, inner sep=.2cm, label={below:Teacher Model\\(Full Precision)}] {};
\end{scope}







\node (stu1) [layer1]  at (7.5,0) {};
\node (stu2) [layer2]  at (7,0) {};
\node (stu3) [layer3]  at (6.5,0) {};
\node (stu4) [layer4]  at (6,0) {};

\path[yshift=1.5cm, xshift=-0.5cm] node[hidden neuron] (H1) at (5.75,-0.5 cm) {};
\path[yshift=1.5cm, xshift=-0.5cm]node[hidden neuron] (H2) at (5.75,-1 cm) {};
\path[yshift=1.5cm, xshift=-0.5cm] node[hidden neuron] (H3) at (5.75,-1.5 cm) {};
\path[yshift=1.5cm, xshift=-0.5cm]node[hidden neuron] (H4) at (5.75,-2 cm) {};
\path[yshift=1.5cm, xshift=-0.5cm] node[hidden neuron] (H5) at (5.75,-2.5 cm) {};

\path[yshift=1.5cm, xshift=-0.5cm] node[output neuron] (O1) at (5.25,-1 cm) {};
\path[yshift=1.5cm, xshift=-0.5cm] node[output neuron] (O2) at (5.25,-1.5 cm) {};
\path[yshift=1.5cm, xshift=-0.5cm] node[output neuron] (O3) at (5.25,-2 cm) {};

\node (out) [output neuron, left of=H3, xshift=1cm] {};
\begin{scope}[on background layer]
    \node (student) [fit=(stu1) (stu2) (stu3) (stu4) (H1) (H2) (H3) (H4) (H5) (O1) (O2) (O3) (out), fill= gray!20, rounded corners, inner sep=.2cm, label={below:Student Model\\(Binary Precision)}] {};
\end{scope}


\node (x_recon) [draw, left of=out] {$f_{student}(\mathcal{X})$};
\node (y_labels) [draw, right of=out1] {$f_{teacher}(\mathcal{X})$};

\node (classification_loss) [draw, left of=x_recon, xshift=-40] {Knowledge Distillation Loss\\$Loss_{KD}$ ($f_{student}, f_{teacher}$)};



\draw[thick,->] ([xshift=0.7cm] classification_loss.south) |- node[anchor=north, yshift=-0cm, xshift=1.5cm] {Weight Update\\(Backpropagation)} ([yshift=-1cm]student.west);
\draw[thick,->] (teacher.east) -- node[anchor=north] {} (y_labels.west);
\draw[thick,->] (y_labels.east) -- node[anchor=north] {} (classification_loss.west);
\draw[thick,->] (x_recon.west) |- node[anchor=north] {} (classification_loss.east);
\draw[thick,->] (student.west) -- node[anchor=north] {} (x_recon);

\end{tikzpicture}
}
\caption{\underline{\textbf{Improving the Binary Model Performance.}} The full precision model is used as a teacher model and the loss function uses the soft labels of the teacher as the target to train and update the Binary Model's performance. The resultant Binary model has a higher test accuracy trained using this fashion at the cost of a small increase in Membership Inference accuracy.}
\label{fig:advclassifier}
\end{figure}


\section{Evaluation}\label{evaluation}

Difficult to train binarized model~\cite{AAAI1714619}

\begin{table}[!htb]
\begin{center}
\renewcommand\arraystretch{1.5}
\fontsize{6.7pt}{6.7pt}\selectfont
\begin{tabular}{|c|c|c|c|c|}
\hline
\multicolumn{5}{|c|}{\textbf{CIFAR10}} \\
\hline
\multicolumn{2}{|c|}{\textbf{Architecture}} & \textbf{Train}  & \textbf{Test}  & \textbf{Inference}  \\
 \multicolumn{2}{|c|}{} & \textbf{Accuracy} & \textbf{Accuracy} & \textbf{Accuracy}  \\
\hline
\multirow{2}{*}{NiN} & Full Precision & 98.16\% & 86.16\% & \cellcolor{red!25}56.69\% \\
& Binary Precision & 81.93\% & 78.74\% & \cellcolor{green!25}51.76\% \\
\hline
\multirow{2}{*}{AlexNet} & Full Precision & 97.86\% & 80.34\% & \cellcolor{red!25}60.40\% \\
& Binary Precision & 68.62\% & 66.8\% & \cellcolor{green!25}51.40\% \\
\hline
\multirow{2}{*}{VGG16} & Full Precision & 99.58\% & 88.95\% & \cellcolor{red!25}58.70\%\\
& Binary Precision & 79.67\% & 74.64\% & \cellcolor{green!25}52.65\%\\
\hline
\end{tabular}
\end{center}
\caption{Reducing the precision of models lowers the membership privacy leakage through membership inference attacks but at the cost of accuracy}
\label{fmnist_quantize}
\end{table}


%\pgfplotsset{footnotesize,height=5.5cm,width=0.35\textwidth}
% \pgfplotsset{footnotesize,samples=10}

\begin{figure*}[b]
\begin{center}% note that \centering uses less vspace...
\resizebox{2\columnwidth}{!}{%
\begin{tabular}{lllll}


\begin{tikzpicture}
\begin{axis}[title={(a)},
title style={at={(0.5,0)},anchor=north,yshift=-35},
ylabel=Confidence Score,
xlabel=Classes,
legend style={font=\large},
legend pos =  north east,
ybar=5pt,% configures ‘bar shift’
bar width=6pt,
xtick={0,1,2,3,4,5,6,7,8,9},
ymin=0
]

\addplot
coordinates {(0,0.0010145490523427725) (1,0.00030558116850443184) (2,0.0008154626120813191) (3,0.9787409901618958) (4,0.00029880856163799763) (5,0.010064681060612202) (6,0.00022722511494066566) (7,0.007986118085682392) (8,0.00020130971097387373) (9,0.00034515938023105264)};
\addplot
coordinates {(0,0.06327114254236221) (1,0.025417674332857132) (2,0.011714407242834568) (3,0.5831579566001892) (4,0.04046289622783661) (5,0.02152826450765133) (6,0.00799587182700634) (7,0.1768314242362976) (8,0.021124042570590973) (9,0.04849636182188988)};

\legend{Member,Non-Member}
\end{axis}
\end{tikzpicture} &


%
\begin{tikzpicture}
\begin{axis}[title={(b)},
title style={at={(0.5,0)},anchor=north,yshift=-35},
ylabel=Confidence Score,
xlabel=Classes,
xtick={0,1,2,3,4,5,6,7,8,9},
legend style={font=\large},
legend pos =  north east,
ybar=5pt,% configures ‘bar shift’
bar width=6pt,
ymin=0
]

\addplot
coordinates {(0,0.07000309228897095) (1,0.017400363460183144) (2,0.019030457362532616) (3,0.29907193779945374) (4,0.10515590757131577) (5,0.22833536565303802) (6,0.04266407713294029) (7,0.04584269970655441) (8,0.11086355894804001) (9,0.061632607132196426)};
\addplot
coordinates {(0,0.12253900617361069) (1,0.0015809950418770313) (2,0.27642616629600525) (3,0.07850959151983261) (4,0.2582301199436188) (5,0.14231687784194946) (6,0.004606064409017563) (7,0.10550655424594879) (8,0.005763859022408724) (9,0.004520699847489595)};

\legend{Member,Non-Member}
\end{axis}
\end{tikzpicture} &





%
\begin{tikzpicture}
\begin{axis}[title={(c)},
title style={at={(0.5,0)},anchor=north,yshift=-35},
line width=1.0pt,
legend style={font=\small},
legend pos =  north east,
legend entries={Full Precision, Binarized, Distilled Binarized},
ylabel={Loss},
xmin=0,
xmax=360,
xlabel={Number of Iterations},
% extra x ticks={1,10,...,400},
% extra y ticks={0,0.5,...,10},
% extra y tick labels={},
% extra x tick labels={},
% extra x tick style={grid=major},
% extra y tick style={grid=major},
grid=major
]
\addplot[
    color=black,
    solid,
    smooth
    ]
    coordinates {
    ( 1 , 2.302968 )
    ( 2 , 1.939443 )
    ( 3 , 1.667299 )
    ( 4 , 1.652660 )
    ( 5 , 1.455838 )
    ( 6 , 1.559521 )
    ( 7 , 1.526025 )
    ( 8 , 1.260517 )
    ( 9 , 1.503018 )
    ( 10 , 1.285601 )
    ( 11 , 1.384281 )
    ( 12 , 1.286194 )
    ( 13 , 1.191418 )
    ( 14 , 1.079983 )
    ( 15 , 1.039906 )
    ( 16 , 1.188020 )
    ( 17 , 0.963824 )
    ( 18 , 0.943251 )
    ( 19 , 1.289535 )
    ( 20 , 1.082280 )
    ( 21 , 1.037991 )
    ( 22 , 0.799126 )
    ( 23 , 0.983668 )
    ( 24 , 0.966282 )
    ( 25 , 0.922792 )
    ( 26 , 0.765660 )
    ( 27 , 0.803365 )
    ( 28 , 0.850827 )
    ( 29 , 0.893062 )
    ( 30 , 1.063586 )
    ( 31 , 0.948933 )
    ( 32 , 0.901259 )
    ( 33 , 0.895251 )
    ( 34 , 0.794324 )
    ( 35 , 0.769924 )
    ( 36 , 0.899247 )
    ( 37 , 0.787751 )
    ( 38 , 0.831077 )
    ( 39 , 0.962390 )
    ( 40 , 0.821883 )
    ( 41 , 0.863004 )
    ( 42 , 0.818929 )
    ( 43 , 0.880331 )
    ( 44 , 0.676965 )
    ( 45 , 0.891206 )
    ( 46 , 0.876807 )
    ( 47 , 0.854072 )
    ( 48 , 0.860776 )
    ( 49 , 0.631650 )
    ( 50 , 0.662787 )
    ( 51 , 0.630878 )
    ( 52 , 0.778465 )
    ( 53 , 0.900262 )
    ( 54 , 0.731535 )
    ( 55 , 0.845547 )
    ( 56 , 0.785181 )
    ( 57 , 0.748590 )
    ( 58 , 0.986032 )
    ( 59 , 0.774422 )
    ( 60 , 0.691421 )
    ( 61 , 0.551147 )
    ( 62 , 0.668453 )
    ( 63 , 0.759031 )
    ( 64 , 0.673677 )
    ( 65 , 0.512989 )
    ( 66 , 0.796980 )
    ( 67 , 0.681779 )
    ( 68 , 0.851441 )
    ( 69 , 1.030697 )
    ( 70 , 0.751324 )
    ( 71 , 0.606460 )
    ( 72 , 0.733667 )
    ( 73 , 0.701998 )
    ( 74 , 0.680291 )
    ( 75 , 0.707206 )
    ( 76 , 0.645020 )
    ( 77 , 0.566519 )
    ( 78 , 0.622747 )
    ( 79 , 0.651569 )
    ( 80 , 0.560384 )
    ( 81 , 0.663585 )
    ( 82 , 0.648072 )
    ( 83 , 0.676273 )
    ( 84 , 0.701089 )
    ( 85 , 0.797614 )
    ( 86 , 0.665663 )
    ( 87 , 0.657626 )
    ( 88 , 0.592905 )
    ( 89 , 0.687644 )
    ( 90 , 0.712525 )
    ( 91 , 0.533854 )
    ( 92 , 0.557828 )
    ( 93 , 0.599991 )
    ( 94 , 0.673825 )
    ( 95 , 0.575448 )
    ( 96 , 0.526772 )
    ( 97 , 0.505667 )
    ( 98 , 0.643386 )
    ( 99 , 0.620009 )
    ( 100 , 0.610189 )
    ( 101 , 0.647692 )
    ( 102 , 0.539143 )
    ( 103 , 0.667260 )
    ( 104 , 0.489376 )
    ( 105 , 0.518314 )
    ( 106 , 0.676982 )
    ( 107 , 0.727331 )
    ( 108 , 0.581430 )
    ( 109 , 0.618056 )
    ( 110 , 0.500167 )
    ( 111 , 0.785589 )
    ( 112 , 0.399176 )
    ( 113 , 0.447618 )
    ( 114 , 0.475429 )
    ( 115 , 0.570146 )
    ( 116 , 0.544356 )
    ( 117 , 0.555152 )
    ( 118 , 0.546433 )
    ( 119 , 0.542349 )
    ( 120 , 0.577833 )
    ( 121 , 0.639318 )
    ( 122 , 0.567917 )
    ( 123 , 0.517693 )
    ( 124 , 0.460981 )
    ( 125 , 0.593507 )
    ( 126 , 0.570267 )
    ( 127 , 0.597812 )
    ( 128 , 0.598578 )
    ( 129 , 0.478810 )
    ( 130 , 0.586665 )
    ( 131 , 0.579736 )
    ( 132 , 0.659663 )
    ( 133 , 0.538597 )
    ( 134 , 0.549711 )
    ( 135 , 0.537823 )
    ( 136 , 0.572637 )
    ( 137 , 0.534132 )
    ( 138 , 0.570913 )
    ( 139 , 0.422553 )
    ( 140 , 0.488932 )
    ( 141 , 0.504971 )
    ( 142 , 0.423515 )
    ( 143 , 0.614434 )
    ( 144 , 0.564107 )
    ( 145 , 0.541557 )
    ( 146 , 0.744249 )
    ( 147 , 0.522557 )
    ( 148 , 0.606243 )
    ( 149 , 0.458936 )
    ( 150 , 0.558472 )
    ( 151 , 0.581740 )
    ( 152 , 0.556220 )
    ( 153 , 0.470633 )
    ( 154 , 0.493428 )
    ( 155 , 0.524618 )
    ( 156 , 0.601933 )
    ( 157 , 0.460702 )
    ( 158 , 0.542096 )
    ( 159 , 0.583347 )
    ( 160 , 0.778570 )
    ( 161 , 0.497549 )
    ( 162 , 0.533701 )
    ( 163 , 0.603471 )
    ( 164 , 0.397012 )
    ( 165 , 0.583966 )
    ( 166 , 0.417388 )
    ( 167 , 0.471393 )
    ( 168 , 0.520279 )
    ( 169 , 0.520528 )
    ( 170 , 0.437146 )
    ( 171 , 0.482065 )
    ( 172 , 0.469343 )
    ( 173 , 0.539111 )
    ( 174 , 0.408798 )
    ( 175 , 0.585470 )
    ( 176 , 0.532898 )
    ( 177 , 0.415162 )
    ( 178 , 0.476431 )
    ( 179 , 0.477853 )
    ( 180 , 0.486712 )
    ( 181 , 0.538571 )
    ( 182 , 0.526860 )
    ( 183 , 0.590460 )
    ( 184 , 0.575449 )
    ( 185 , 0.451290 )
    ( 186 , 0.401073 )
    ( 187 , 0.596629 )
    ( 188 , 0.545198 )
    ( 189 , 0.532833 )
    ( 190 , 0.382534 )
    ( 191 , 0.532536 )
    ( 192 , 0.539632 )
    ( 193 , 0.492966 )
    ( 194 , 0.496653 )
    ( 195 , 0.291686 )
    ( 196 , 0.359933 )
    ( 197 , 0.558284 )
    ( 198 , 0.393615 )
    ( 199 , 0.546782 )
    ( 200 , 0.507714 )
    ( 201 , 0.441790 )
    ( 202 , 0.694662 )
    ( 203 , 0.387495 )
    ( 204 , 0.539195 )
    ( 205 , 0.534799 )
    ( 206 , 0.460438 )
    ( 207 , 0.412882 )
    ( 208 , 0.439468 )
    ( 209 , 0.461498 )
    ( 210 , 0.484441 )
    ( 211 , 0.371391 )
    ( 212 , 0.505832 )
    ( 213 , 0.518648 )
    ( 214 , 0.544090 )
    ( 215 , 0.680713 )
    ( 216 , 0.751645 )
    ( 217 , 0.478645 )
    ( 218 , 0.442678 )
    ( 219 , 0.387577 )
    ( 220 , 0.538398 )
    ( 221 , 0.543274 )
    ( 222 , 0.495477 )
    ( 223 , 0.428225 )
    ( 224 , 0.311347 )
    ( 225 , 0.386528 )
    ( 226 , 0.415797 )
    ( 227 , 0.386841 )
    ( 228 , 0.512556 )
    ( 229 , 0.467811 )
    ( 230 , 0.469653 )
    ( 231 , 0.497145 )
    ( 232 , 0.386810 )
    ( 233 , 0.365984 )
    ( 234 , 0.370858 )
    ( 235 , 0.334157 )
    ( 236 , 0.421482 )
    ( 237 , 0.437231 )
    ( 238 , 0.533400 )
    ( 239 , 0.363779 )
    ( 240 , 0.499905 )
    ( 241 , 0.448234 )
    ( 242 , 0.381877 )
    ( 243 , 0.457827 )
    ( 244 , 0.508316 )
    ( 245 , 0.432367 )
    ( 246 , 0.550766 )
    ( 247 , 0.567483 )
    ( 248 , 0.600920 )
    ( 249 , 0.619330 )
    ( 250 , 0.501078 )
    ( 251 , 0.563693 )
    ( 252 , 0.420956 )
    ( 253 , 0.452368 )
    ( 254 , 0.529415 )
    ( 255 , 0.413688 )
    ( 256 , 0.507326 )
    ( 257 , 0.534332 )
    ( 258 , 0.340995 )
    ( 259 , 0.533992 )
    ( 260 , 0.445496 )
    ( 261 , 0.371866 )
    ( 262 , 0.357937 )
    ( 263 , 0.387621 )
    ( 264 , 0.290543 )
    ( 265 , 0.351936 )
    ( 266 , 0.467249 )
    ( 267 , 0.429069 )
    ( 268 , 0.564726 )
    ( 269 , 0.309757 )
    ( 270 , 0.382020 )
    ( 271 , 0.423434 )
    ( 272 , 0.425730 )
    ( 273 , 0.322237 )
    ( 274 , 0.597025 )
    ( 275 , 0.560745 )
    ( 276 , 0.445284 )
    ( 277 , 0.378794 )
    ( 278 , 0.276946 )
    ( 279 , 0.282341 )
    ( 280 , 0.366811 )
    ( 281 , 0.478299 )
    ( 282 , 0.375445 )
    ( 283 , 0.480307 )
    ( 284 , 0.493695 )
    ( 285 , 0.336665 )
    ( 286 , 0.462213 )
    ( 287 , 0.346743 )
    ( 288 , 0.541349 )
    ( 289 , 0.522342 )
    ( 290 , 0.514670 )
    ( 291 , 0.430172 )
    ( 292 , 0.289797 )
    ( 293 , 0.371195 )
    ( 294 , 0.405113 )
    ( 295 , 0.357373 )
    ( 296 , 0.603785 )
    ( 297 , 0.321793 )
    ( 298 , 0.506746 )
    ( 299 , 0.432704 )
    ( 300 , 0.405093 )
    ( 301 , 0.403066 )
    ( 302 , 0.504766 )
    ( 303 , 0.473787 )
    ( 304 , 0.425059 )
    ( 305 , 0.450767 )
    ( 306 , 0.582399 )
    ( 307 , 0.300974 )
    ( 308 , 0.433699 )
    ( 309 , 0.439344 )
    ( 310 , 0.394054 )
    ( 311 , 0.458976 )
    ( 312 , 0.482490 )
    ( 313 , 0.442764 )
    ( 314 , 0.467489 )
    ( 315 , 0.462829 )
    ( 316 , 0.372411 )
    ( 317 , 0.298418 )
    ( 318 , 0.353107 )
    ( 319 , 0.452360 )
    ( 320 , 0.359912 )
    ( 321 , 0.388992 )
    ( 322 , 0.428388 )
    ( 323 , 0.456512 )
    ( 324 , 0.380230 )
    ( 325 , 0.357444 )
    ( 326 , 0.403996 )
    ( 327 , 0.361251 )
    ( 328 , 0.344514 )
    ( 329 , 0.381908 )
    ( 330 , 0.289464 )
    ( 331 , 0.320182 )
    ( 332 , 0.425570 )
    ( 333 , 0.448104 )
    ( 334 , 0.466504 )
    ( 335 , 0.340016 )
    ( 336 , 0.430965 )
    ( 337 , 0.396781 )
    ( 338 , 0.356925 )
    ( 339 , 0.267577 )
    ( 340 , 0.468913 )
    ( 341 , 0.233512 )
    ( 342 , 0.466996 )
    ( 343 , 0.332996 )
    ( 344 , 0.367131 )
    ( 345 , 0.389329 )
    ( 346 , 0.330957 )
    ( 347 , 0.378960 )
    ( 348 , 0.278310 )
    ( 349 , 0.290548 )
    ( 350 , 0.316687 )
    ( 351 , 0.388542 )
    ( 352 , 0.388174 )
    ( 353 , 0.279368 )
    ( 354 , 0.412089 )
    ( 355 , 0.335098 )
    ( 356 , 0.229976 )
    ( 357 , 0.449180 )
    ( 358 , 0.374953 )
    ( 359 , 0.321231 )
    ( 360 , 0.348795 )
      };

      \addplot[
          color=gray,
          solid,
          smooth
          ]
          coordinates {
          ( 1 , 2.346761 )
  ( 2 , 1.926266 )
  ( 3 , 1.885543 )
  ( 4 , 1.695635 )
  ( 5 , 1.828619 )
  ( 6 , 1.763135 )
  ( 7 , 1.675170 )
  ( 8 , 1.648822 )
  ( 9 , 1.501599 )
  ( 10 , 1.517566 )
  ( 11 , 1.694481 )
  ( 12 , 1.630272 )
  ( 13 , 1.482632 )
  ( 14 , 1.509774 )
  ( 15 , 1.607969 )
  ( 16 , 1.573090 )
  ( 17 , 1.538904 )
  ( 18 , 1.573854 )
  ( 19 , 1.506782 )
  ( 20 , 1.384800 )
  ( 21 , 1.485137 )
  ( 22 , 1.445732 )
  ( 23 , 1.566089 )
  ( 24 , 1.400355 )
  ( 25 , 1.410617 )
  ( 26 , 1.602641 )
  ( 27 , 1.363793 )
  ( 28 , 1.374451 )
  ( 29 , 1.327962 )
  ( 30 , 1.259157 )
  ( 31 , 1.158507 )
  ( 32 , 1.471694 )
  ( 33 , 1.281064 )
  ( 34 , 1.395380 )
  ( 35 , 1.413245 )
  ( 36 , 1.364599 )
  ( 37 , 1.446778 )
  ( 38 , 1.205966 )
  ( 39 , 1.332203 )
  ( 40 , 1.474646 )
  ( 41 , 1.530147 )
  ( 42 , 1.370293 )
  ( 43 , 1.436183 )
  ( 44 , 1.444681 )
  ( 45 , 1.332167 )
  ( 46 , 1.484209 )
  ( 47 , 1.465831 )
  ( 48 , 1.305787 )
  ( 49 , 1.380785 )
  ( 50 , 1.305335 )
  ( 51 , 1.252715 )
  ( 52 , 1.414409 )
  ( 53 , 1.208195 )
  ( 54 , 1.097216 )
  ( 55 , 1.304714 )
  ( 56 , 1.456459 )
  ( 57 , 1.040485 )
  ( 58 , 1.141915 )
  ( 59 , 1.240699 )
  ( 60 , 1.379954 )
  ( 61 , 1.174325 )
  ( 62 , 1.360094 )
  ( 63 , 1.193203 )
  ( 64 , 1.246328 )
  ( 65 , 1.128607 )
  ( 66 , 1.297393 )
  ( 67 , 1.293353 )
  ( 68 , 1.207303 )
  ( 69 , 1.274779 )
  ( 70 , 1.260201 )
  ( 71 , 1.205472 )
  ( 72 , 1.257582 )
  ( 73 , 1.209281 )
  ( 74 , 1.320339 )
  ( 75 , 1.081225 )
  ( 76 , 1.379258 )
  ( 77 , 1.403371 )
  ( 78 , 1.173059 )
  ( 79 , 1.210906 )
  ( 80 , 1.212522 )
  ( 81 , 1.256900 )
  ( 82 , 1.261412 )
  ( 83 , 1.132829 )
  ( 84 , 1.261608 )
  ( 85 , 1.130152 )
  ( 86 , 1.307679 )
  ( 87 , 1.242157 )
  ( 88 , 1.203381 )
  ( 89 , 1.197875 )
  ( 90 , 1.127777 )
  ( 91 , 1.099596 )
  ( 92 , 1.218491 )
  ( 93 , 0.959279 )
  ( 94 , 1.230520 )
  ( 95 , 1.063139 )
  ( 96 , 1.248715 )
  ( 97 , 1.083069 )
  ( 98 , 1.019647 )
  ( 99 , 1.012220 )
  ( 100 , 1.250834 )
  ( 101 , 1.113485 )
  ( 102 , 1.065619 )
  ( 103 , 1.023666 )
  ( 104 , 1.063910 )
  ( 105 , 1.067258 )
  ( 106 , 1.032276 )
  ( 107 , 1.165910 )
  ( 108 , 1.197135 )
  ( 109 , 1.079468 )
  ( 110 , 1.176466 )
  ( 111 , 1.049984 )
  ( 112 , 1.325067 )
  ( 113 , 1.198396 )
  ( 114 , 1.209702 )
  ( 115 , 1.122866 )
  ( 116 , 1.246204 )
  ( 117 , 1.130638 )
  ( 118 , 1.070367 )
  ( 119 , 1.258520 )
  ( 120 , 1.100373 )
  ( 121 , 1.169035 )
  ( 122 , 1.103153 )
  ( 123 , 1.187527 )
  ( 124 , 1.249404 )
  ( 125 , 1.039255 )
  ( 126 , 1.043600 )
  ( 127 , 1.163061 )
  ( 128 , 0.866229 )
  ( 129 , 1.100545 )
  ( 130 , 1.150101 )
  ( 131 , 1.129426 )
  ( 132 , 1.129657 )
  ( 133 , 1.159651 )
  ( 134 , 0.838983 )
  ( 135 , 1.136622 )
  ( 136 , 1.221408 )
  ( 137 , 1.139824 )
  ( 138 , 1.213746 )
  ( 139 , 0.979337 )
  ( 140 , 1.232616 )
  ( 141 , 0.885131 )
  ( 142 , 1.174824 )
  ( 143 , 0.979129 )
  ( 144 , 1.055910 )
  ( 145 , 1.090147 )
  ( 146 , 0.904567 )
  ( 147 , 1.113498 )
  ( 148 , 1.125462 )
  ( 149 , 1.049405 )
  ( 150 , 0.985949 )
  ( 151 , 1.139380 )
  ( 152 , 0.959967 )
  ( 153 , 1.000041 )
  ( 154 , 1.263282 )
  ( 155 , 0.956478 )
  ( 156 , 0.995327 )
  ( 157 , 1.096886 )
  ( 158 , 1.111590 )
  ( 159 , 1.176757 )
  ( 160 , 0.962864 )
  ( 161 , 0.977689 )
  ( 162 , 0.922844 )
  ( 163 , 1.120963 )
  ( 164 , 1.009984 )
  ( 165 , 1.063984 )
  ( 166 , 1.032832 )
  ( 167 , 0.987173 )
  ( 168 , 1.035523 )
  ( 169 , 0.903309 )
  ( 170 , 1.083587 )
  ( 171 , 1.178625 )
  ( 172 , 1.068447 )
  ( 173 , 0.984077 )
  ( 174 , 1.152232 )
  ( 175 , 0.932497 )
  ( 176 , 0.932060 )
  ( 177 , 0.994724 )
  ( 178 , 1.047627 )
  ( 179 , 1.068406 )
  ( 180 , 1.097718 )
  ( 181 , 0.946308 )
  ( 182 , 0.981098 )
  ( 183 , 1.067997 )
  ( 184 , 1.027874 )
  ( 185 , 0.826300 )
  ( 186 , 1.179658 )
  ( 187 , 1.052969 )
  ( 188 , 1.142715 )
  ( 189 , 0.991129 )
  ( 190 , 1.213003 )
  ( 191 , 1.045486 )
  ( 192 , 1.006147 )
  ( 193 , 1.109902 )
  ( 194 , 1.212793 )
  ( 195 , 0.912005 )
  ( 196 , 1.123085 )
  ( 197 , 0.926187 )
  ( 198 , 0.858419 )
  ( 199 , 0.916872 )
  ( 200 , 0.915458 )
  ( 201 , 0.904451 )
  ( 202 , 0.942618 )
  ( 203 , 0.883013 )
  ( 204 , 1.090961 )
  ( 205 , 1.045205 )
  ( 206 , 1.059832 )
  ( 207 , 1.141301 )
  ( 208 , 1.089335 )
  ( 209 , 1.039749 )
  ( 210 , 1.063170 )
  ( 211 , 0.912232 )
  ( 212 , 0.915569 )
  ( 213 , 1.075158 )
  ( 214 , 0.940946 )
  ( 215 , 1.157747 )
  ( 216 , 1.170259 )
  ( 217 , 1.066205 )
  ( 218 , 1.024397 )
  ( 219 , 0.852790 )
  ( 220 , 0.950637 )
  ( 221 , 1.110507 )
  ( 222 , 1.036702 )
  ( 223 , 0.984222 )
  ( 224 , 0.952827 )
  ( 225 , 1.188724 )
  ( 226 , 1.000404 )
  ( 227 , 1.028348 )
  ( 228 , 0.988035 )
  ( 229 , 0.983033 )
  ( 230 , 0.942389 )
  ( 231 , 0.935704 )
  ( 232 , 1.051824 )
  ( 233 , 1.005647 )
  ( 234 , 1.006882 )
  ( 235 , 1.072922 )
  ( 236 , 1.043638 )
  ( 237 , 1.221575 )
  ( 238 , 1.003483 )
  ( 239 , 1.138196 )
  ( 240 , 0.862098 )
  ( 241 , 1.062384 )
  ( 242 , 0.999307 )
  ( 243 , 0.991120 )
  ( 244 , 0.926649 )
  ( 245 , 0.986342 )
  ( 246 , 0.992192 )
  ( 247 , 1.010563 )
  ( 248 , 0.952089 )
  ( 249 , 1.165234 )
  ( 250 , 1.141195 )
  ( 251 , 1.059924 )
  ( 252 , 0.980622 )
  ( 253 , 1.091556 )
  ( 254 , 1.136913 )
  ( 255 , 0.883962 )
  ( 256 , 1.111804 )
  ( 257 , 0.826448 )
  ( 258 , 0.934091 )
  ( 259 , 0.895227 )
  ( 260 , 0.889213 )
  ( 261 , 1.103202 )
  ( 262 , 0.953661 )
  ( 263 , 1.005441 )
  ( 264 , 1.058801 )
  ( 265 , 1.057636 )
  ( 266 , 0.903979 )
  ( 267 , 1.003013 )
  ( 268 , 1.001371 )
  ( 269 , 1.045060 )
  ( 270 , 0.995644 )
  ( 271 , 1.017072 )
  ( 272 , 1.109493 )
  ( 273 , 1.156392 )
  ( 274 , 0.859848 )
  ( 275 , 0.867010 )
  ( 276 , 1.039820 )
  ( 277 , 0.963619 )
  ( 278 , 1.051263 )
  ( 279 , 1.029461 )
  ( 280 , 1.002586 )
  ( 281 , 0.965037 )
  ( 282 , 1.089420 )
  ( 283 , 1.019706 )
  ( 284 , 0.973446 )
  ( 285 , 1.068585 )
  ( 286 , 0.790222 )
  ( 287 , 1.146344 )
  ( 288 , 0.802443 )
  ( 289 , 1.046202 )
  ( 290 , 0.997245 )
  ( 291 , 0.754709 )
  ( 292 , 0.945580 )
  ( 293 , 1.069640 )
  ( 294 , 0.858742 )
  ( 295 , 0.934347 )
  ( 296 , 1.114184 )
  ( 297 , 0.985928 )
  ( 298 , 0.977279 )
  ( 299 , 1.034217 )
  ( 300 , 1.159927 )
  ( 301 , 0.804602 )
  ( 302 , 0.978465 )
  ( 303 , 0.916436 )
  ( 304 , 0.819233 )
  ( 305 , 0.947365 )
  ( 306 , 0.873999 )
  ( 307 , 0.814669 )
  ( 308 , 1.060480 )
  ( 309 , 1.009367 )
  ( 310 , 0.923004 )
  ( 311 , 1.046344 )
  ( 312 , 0.884677 )
  ( 313 , 0.931433 )
  ( 314 , 0.941696 )
  ( 315 , 0.756139 )
  ( 316 , 1.241746 )
  ( 317 , 0.852658 )
  ( 318 , 0.924763 )
  ( 319 , 0.942495 )
  ( 320 , 0.954869 )
  ( 321 , 1.075119 )
  ( 322 , 1.036582 )
  ( 323 , 0.906247 )
  ( 324 , 0.812499 )
  ( 325 , 0.977565 )
  ( 326 , 0.870054 )
  ( 327 , 0.857418 )
  ( 328 , 0.904272 )
  ( 329 , 0.964146 )
  ( 330 , 0.960810 )
  ( 331 , 0.922262 )
  ( 332 , 0.889789 )
  ( 333 , 1.004996 )
  ( 334 , 1.073944 )
  ( 335 , 0.891395 )
  ( 336 , 1.000604 )
  ( 337 , 0.882347 )
  ( 338 , 1.048631 )
  ( 339 , 0.699180 )
  ( 340 , 1.102788 )
  ( 341 , 1.062422 )
  ( 342 , 0.861747 )
  ( 343 , 1.025133 )
  ( 344 , 0.951212 )
  ( 345 , 0.927888 )
  ( 346 , 0.904762 )
  ( 347 , 0.950766 )
  ( 348 , 0.868892 )
  ( 349 , 1.051545 )
  ( 350 , 1.070194 )
  ( 351 , 1.035250 )
  ( 352 , 0.803139 )
  ( 353 , 0.765967 )
  ( 354 , 0.936454 )
  ( 355 , 0.882868 )
  ( 356 , 0.877807 )
  ( 357 , 0.886292 )
  ( 358 , 1.150183 )
  ( 359 , 0.951342 )
  ( 360 , 0.913005 )
            };

  \addplot[
      color=red,
      solid,
      smooth
      ]
      coordinates {
      ( 1 , 2.346761 )
      ( 1 , 2.289508 )
      ( 2 , 1.958497 )
      ( 3 , 1.921034 )
      ( 4 , 1.561351 )
      ( 5 , 1.560222 )
      ( 6 , 1.672546 )
      ( 7 , 1.671921 )
      ( 8 , 1.527745 )
      ( 9 , 1.516091 )
      ( 10 , 1.320806 )
      ( 11 , 1.168530 )
      ( 12 , 1.210074 )
      ( 13 , 1.297815 )
      ( 14 , 1.161482 )
      ( 15 , 1.246961 )
      ( 16 , 1.331731 )
      ( 17 , 1.299713 )
      ( 18 , 1.200435 )
      ( 19 , 1.183215 )
      ( 20 , 1.105063 )
      ( 21 , 1.109861 )
      ( 22 , 1.254844 )
      ( 23 , 1.108233 )
      ( 24 , 0.924076 )
      ( 25 , 1.244691 )
      ( 26 , 1.063551 )
      ( 27 , 0.970651 )
      ( 28 , 0.976295 )
      ( 29 , 0.913169 )
      ( 30 , 1.119265 )
      ( 31 , 1.173765 )
      ( 32 , 0.947561 )
      ( 33 , 1.054491 )
      ( 34 , 1.046588 )
      ( 35 , 1.206818 )
      ( 36 , 0.882458 )
      ( 37 , 0.983017 )
      ( 38 , 1.080579 )
      ( 39 , 1.086890 )
      ( 40 , 0.842886 )
      ( 41 , 1.221012 )
      ( 42 , 1.075716 )
      ( 43 , 0.899074 )
      ( 44 , 1.293818 )
      ( 45 , 0.811225 )
      ( 46 , 0.981191 )
      ( 47 , 1.177047 )
      ( 48 , 1.032480 )
      ( 49 , 0.981890 )
      ( 50 , 1.005604 )
      ( 51 , 0.828696 )
      ( 52 , 1.057695 )
      ( 53 , 0.824240 )
      ( 54 , 1.040227 )
      ( 55 , 0.879299 )
      ( 56 , 0.971962 )
      ( 57 , 0.997197 )
      ( 58 , 0.886545 )
      ( 59 , 1.017593 )
      ( 60 , 0.762022 )
      ( 61 , 1.199266 )
      ( 62 , 0.736034 )
      ( 63 , 0.687340 )
      ( 64 , 0.962291 )
      ( 65 , 0.905756 )
      ( 66 , 0.875405 )
      ( 67 , 0.986668 )
      ( 68 , 0.885548 )
      ( 69 , 0.877220 )
      ( 70 , 0.914856 )
      ( 71 , 0.682138 )
      ( 72 , 0.919164 )
      ( 73 , 1.070877 )
      ( 74 , 0.855854 )
      ( 75 , 0.627857 )
      ( 76 , 0.856239 )
      ( 77 , 0.792077 )
      ( 78 , 0.823261 )
      ( 79 , 0.830847 )
      ( 80 , 0.857799 )
      ( 81 , 0.892394 )
      ( 82 , 0.966335 )
      ( 83 , 0.803680 )
      ( 84 , 0.914873 )
      ( 85 , 0.908078 )
      ( 86 , 0.798237 )
      ( 87 , 0.752717 )
      ( 88 , 0.869953 )
      ( 89 , 0.796312 )
      ( 90 , 0.844140 )
      ( 91 , 0.699578 )
      ( 92 , 0.830843 )
      ( 93 , 0.896852 )
      ( 94 , 0.871961 )
      ( 95 , 0.915021 )
      ( 96 , 0.821148 )
      ( 97 , 0.817516 )
      ( 98 , 0.820520 )
      ( 99 , 0.750622 )
      ( 100 , 0.976288 )
      ( 101 , 0.791571 )
      ( 102 , 0.738218 )
      ( 103 , 0.854439 )
      ( 104 , 0.610627 )
      ( 105 , 0.990846 )
      ( 106 , 0.649290 )
      ( 107 , 0.813209 )
      ( 108 , 1.002776 )
      ( 109 , 0.739490 )
      ( 110 , 0.742654 )
      ( 111 , 0.952149 )
      ( 112 , 0.742725 )
      ( 113 , 0.953895 )
      ( 114 , 0.866470 )
      ( 115 , 0.899595 )
      ( 116 , 0.905452 )
      ( 117 , 0.761735 )
      ( 118 , 0.759309 )
      ( 119 , 0.824847 )
      ( 120 , 0.632895 )
      ( 121 , 0.629622 )
      ( 122 , 0.782615 )
      ( 123 , 0.652874 )
      ( 124 , 0.728099 )
      ( 125 , 0.806889 )
      ( 126 , 0.848280 )
      ( 127 , 0.685887 )
      ( 128 , 0.781918 )
      ( 129 , 0.568329 )
      ( 130 , 0.653542 )
      ( 131 , 0.681325 )
      ( 132 , 0.587069 )
      ( 133 , 0.728819 )
      ( 134 , 0.688508 )
      ( 135 , 0.655295 )
      ( 136 , 0.647998 )
      ( 137 , 0.887438 )
      ( 138 , 0.804118 )
      ( 139 , 0.681075 )
      ( 140 , 0.801215 )
      ( 141 , 0.765510 )
      ( 142 , 0.810749 )
      ( 143 , 0.832472 )
      ( 144 , 0.954358 )
      ( 145 , 0.698934 )
      ( 146 , 0.696781 )
      ( 147 , 0.720781 )
      ( 148 , 0.611648 )
      ( 149 , 0.570984 )
      ( 150 , 0.805634 )
      ( 151 , 0.698532 )
      ( 152 , 0.715090 )
      ( 153 , 0.752225 )
      ( 154 , 0.854783 )
      ( 155 , 0.743349 )
      ( 156 , 0.737030 )
      ( 157 , 0.898713 )
      ( 158 , 0.668182 )
      ( 159 , 0.672265 )
      ( 160 , 0.889392 )
      ( 161 , 0.825691 )
      ( 162 , 0.970438 )
      ( 163 , 0.662543 )
      ( 164 , 0.723655 )
      ( 165 , 0.652735 )
      ( 166 , 0.684006 )
      ( 167 , 0.844469 )
      ( 168 , 0.903263 )
      ( 169 , 0.756717 )
      ( 170 , 0.695948 )
      ( 171 , 0.658278 )
      ( 172 , 0.890274 )
      ( 173 , 0.604330 )
      ( 174 , 0.846531 )
      ( 175 , 0.740486 )
      ( 176 , 0.697799 )
      ( 177 , 0.732877 )
      ( 178 , 0.674738 )
      ( 179 , 0.642503 )
      ( 180 , 0.539443 )
      ( 181 , 0.657330 )
      ( 182 , 0.605968 )
      ( 183 , 0.715463 )
      ( 184 , 0.682130 )
      ( 185 , 0.617121 )
      ( 186 , 0.661775 )
      ( 187 , 0.788635 )
      ( 188 , 0.588687 )
      ( 189 , 0.700439 )
      ( 190 , 0.697120 )
      ( 191 , 0.781461 )
      ( 192 , 0.660160 )
      ( 193 , 0.770010 )
      ( 194 , 0.800414 )
      ( 195 , 0.835379 )
      ( 196 , 0.684457 )
      ( 197 , 0.623300 )
      ( 198 , 0.611837 )
      ( 199 , 0.672864 )
      ( 200 , 0.579889 )
      ( 201 , 0.766305 )
      ( 202 , 0.661602 )
      ( 203 , 0.654130 )
      ( 204 , 0.659236 )
      ( 205 , 0.676831 )
      ( 206 , 0.763144 )
      ( 207 , 0.783112 )
      ( 208 , 0.779393 )
      ( 209 , 0.611283 )
      ( 210 , 0.660481 )
      ( 211 , 0.622174 )
      ( 212 , 0.739116 )
      ( 213 , 0.736537 )
      ( 214 , 0.717488 )
      ( 215 , 0.603474 )
      ( 216 , 0.720956 )
      ( 217 , 0.895162 )
      ( 218 , 0.583987 )
      ( 219 , 0.885873 )
      ( 220 , 0.865381 )
      ( 221 , 0.591229 )
      ( 222 , 0.669389 )
      ( 223 , 0.612504 )
      ( 224 , 0.592590 )
      ( 225 , 0.660891 )
      ( 226 , 0.677134 )
      ( 227 , 0.708002 )
      ( 228 , 0.798415 )
      ( 229 , 0.613641 )
      ( 230 , 0.610063 )
      ( 231 , 0.912058 )
      ( 232 , 0.711359 )
      ( 233 , 0.578273 )
      ( 234 , 0.661869 )
      ( 235 , 0.692462 )
      ( 236 , 0.670032 )
      ( 237 , 0.670372 )
      ( 238 , 0.722965 )
      ( 239 , 0.490983 )
      ( 240 , 0.599413 )
      ( 241 , 0.564405 )
      ( 242 , 0.720429 )
      ( 243 , 0.738146 )
      ( 244 , 0.895175 )
      ( 245 , 0.716247 )
      ( 246 , 0.657736 )
      ( 247 , 0.809308 )
      ( 248 , 0.673177 )
      ( 249 , 0.861484 )
      ( 250 , 0.572226 )
      ( 251 , 0.659255 )
      ( 252 , 0.596616 )
      ( 253 , 0.746455 )
      ( 254 , 0.621354 )
      ( 255 , 0.668629 )
      ( 256 , 0.692351 )
      ( 257 , 0.772648 )
      ( 258 , 0.658080 )
      ( 259 , 0.568799 )
      ( 260 , 0.655683 )
      ( 261 , 0.536369 )
      ( 262 , 0.655911 )
      ( 263 , 0.821214 )
      ( 264 , 0.690711 )
      ( 265 , 0.666788 )
      ( 266 , 0.769021 )
      ( 267 , 0.603920 )
      ( 268 , 0.538827 )
      ( 269 , 0.586855 )
      ( 270 , 0.653802 )
      ( 271 , 0.729698 )
      ( 272 , 0.662506 )
      ( 273 , 0.573307 )
      ( 274 , 0.698765 )
      ( 275 , 0.571538 )
      ( 276 , 0.560311 )
      ( 277 , 0.653228 )
      ( 278 , 0.674414 )
      ( 279 , 0.609833 )
      ( 280 , 0.754565 )
      ( 281 , 0.590132 )
      ( 282 , 0.637046 )
      ( 283 , 0.533268 )
      ( 284 , 0.690426 )
      ( 285 , 0.608395 )
      ( 286 , 0.782570 )
      ( 287 , 0.586600 )
      ( 288 , 0.720135 )
      ( 289 , 0.644524 )
      ( 290 , 0.636831 )
      ( 291 , 0.682296 )
      ( 292 , 0.644609 )
      ( 293 , 0.619640 )
      ( 294 , 0.721622 )
      ( 295 , 0.595761 )
      ( 296 , 0.669771 )
      ( 297 , 0.672078 )
      ( 298 , 0.702717 )
      ( 299 , 0.700702 )
      ( 300 , 0.623715 )
      ( 301 , 0.752632 )
      ( 302 , 0.772183 )
      ( 303 , 0.574867 )
      ( 304 , 0.694724 )
      ( 305 , 0.579246 )
      ( 306 , 0.763784 )
      ( 307 , 0.584890 )
      ( 308 , 0.568738 )
      ( 309 , 0.728641 )
      ( 310 , 0.555155 )
      ( 311 , 0.508444 )
      ( 312 , 0.696937 )
      ( 313 , 0.616997 )
      ( 314 , 0.615176 )
      ( 315 , 0.667652 )
      ( 316 , 0.675299 )
      ( 317 , 0.810867 )
      ( 318 , 0.603052 )
      ( 319 , 0.418517 )
      ( 320 , 0.652455 )
      ( 321 , 0.513917 )
      ( 322 , 0.730091 )
      ( 323 , 0.720278 )
      ( 324 , 0.729959 )
      ( 325 , 0.768966 )
      ( 326 , 0.547639 )
      ( 327 , 0.893457 )
      ( 328 , 0.526425 )
      ( 329 , 0.687370 )
      ( 330 , 0.572277 )
      ( 331 , 0.553026 )
      ( 332 , 0.636406 )
      ( 333 , 0.578816 )
      ( 334 , 0.692453 )
      ( 335 , 0.641644 )
      ( 336 , 0.629935 )
      ( 337 , 0.588312 )
      ( 338 , 0.677901 )
      ( 339 , 0.682241 )
      ( 340 , 0.924726 )
      ( 341 , 0.736128 )
      ( 342 , 0.794371 )
      ( 343 , 0.559741 )
      ( 344 , 0.550659 )
      ( 345 , 0.842112 )
      ( 346 , 0.594323 )
      ( 347 , 0.587956 )
      ( 348 , 0.653175 )
      ( 349 , 0.598934 )
      ( 350 , 0.627735 )
      ( 351 , 0.535564 )
      ( 352 , 0.593137 )
      ( 353 , 0.614303 )
      ( 354 , 0.503203 )
      ( 355 , 0.654863 )
      ( 356 , 0.783821 )
      ( 357 , 0.581702 )
      ( 358 , 0.599832 )
      ( 359 , 0.742905 )
      ( 360 , 0.518803 )
        };

\end{axis}
\end{tikzpicture}
\end{tabular}
}
\caption{(a) Confidence scores are distinguishable between train and test data records in undefended models making them vulnerable to membership inference attacks, (b) \method\hspace{0.02in} models have indistinguishable confidence scores, (c) Loss trajectories in Phase I are higher than Phase II indicating the improvement in accuracy.}
\label{fig:loss}
\end{center}
\end{figure*}



\begin{table}[!htb]
\begin{center}
\renewcommand\arraystretch{1.5}
\fontsize{6.7pt}{6.7pt}\selectfont
\begin{tabular}{|c|c|c|c|c|c|}
\hline
\textbf{Teacher} & \textbf{Student} & \textbf{Train}  & \textbf{Test}  & \textbf{Inference}  \\
&  & \textbf{Accuracy} & \textbf{Accuracy} & \textbf{Accuracy}  \\
\hline
Binary NiN & None & 81.93\% & 78.74\% & 51.76\% \\
Binary AlexNet & None & 68.62\% & 66.8\% & 51.40\% \\
Binary VGG16 & None & 79.67\% & 74.64\% & 52.65\%\\
\hline
NiN & Binary NiN & 90.49\% & 83.52\% & 53.90\% \\
AlexNet & Binary AlexNet & 76.79\% & 73.5\% & 51.85\% \\
VGG16 & Binary VGG16 & 89.45\% & 81.58\% & 54.98\%\\
\hline
DenseNet169 & NiN & 92.84\% & 83.71\% & 54.95\%\\
DenseNet169 & AlexNet & 81.87\% & 76.23\% & 53.51\%\\
DenseNet169 & VGG16 & 93.45\% & 85.8\% & 54.17\%\\
\hline
ResNet50 & NiN & 91.74\% & 83.77\% & 54.53\% \\
ResNet50 & AlexNet & 80.12\% & 74.92\% & 53.12\%\\
ResNet50 & VGG16 & 94.23\% & 86.52\% & 54.46\%\\
\hline
\end{tabular}
\end{center}
\caption{CIFAR10 Cross Architecture. Heterogeneous}
\label{kd}
\end{table}


\subsection{Comparison with prior Defenses}

\subsubsection{Baselines}

Several works have explored approaches to defend against membership inference attacks.
These defences have mainly focuessed on improving the model's generalization and reduce overfitting which has been considered as the main cause for leakage through membership inference attacks.


\noindent\textbf{Adversarial Regularization (AdvReg).}~\cite{DBLP:conf/ccs/NasrSH18}


\noindent\textbf{Differential Privacy (DP).}~\cite{Abadi:2016:DLD:2976749.2978318}

\begin{figure}[ht!]
\begin{center}% note that \centering uses less vspace...
\resizebox{\columnwidth}{!}{%
\begin{tikzpicture}
    \begin{axis}[
        footnotesize,
        % set the `width' of the plot to the maximum length ...
        width=\textwidth,
        % ... and use half this length for the `height'
        height=0.5\textwidth,
        % use `data' for the positioning of the `xticks' ...
        xtick=data,
        % ... and use table data for labeling the `xticks'
        xticklabels from table={comparedef.txt}{n},
        % add extra ticks "at the empty entries to add the vertical lines
        extra x ticks={5,11,17,23},
        % this ticks shouldn't be labeled ...
        extra x tick labels={},
        % ... but grid lines should be drawn without the tick lines
        extra x tick style={
            grid=major,
            major tick length=0pt,
        },
        ymin=0,
        ymax=105,
        xlabel={Number of Layers},
        ylabel={Accuracy \%},
        % because of the category labels, shift the `xlabel' a bit down
        xlabel style={
            yshift=-4ex,
            font=\huge,
        },
        ylabel style={
            font=\huge,
        },
        xticklabel style = {font=\large},
        yticklabel style = {font=\large},
        legend style={at={(0.5,-0.15)},anchor=north,legend columns=-1,  font=\huge},
        % adjust `bar width' so it fits your needs ...
        bar width=8pt,
        % ... and with that you also have to adjust the x limits
        enlarge x limits={abs=1},
        % set `clip mode' to `individual' so the category labels aren't clipped away
        clip mode=individual,
    ]

    % plot the "red" ybars
        \addplot [
            ybar,
            draw=red,
            pattern color=red,
            pattern=dots,
        ] table [
            % use just the `coordindex' as x coordinate,
            % the correct labeling is done with `xticklabels from table'
            x expr=\coordindex,
            y=pFA,
        ] {comparedef.txt};

    % plot the "blue" ybars
        \addplot [
            ybar,
            draw=blue,
            pattern color=blue,
            pattern=north east lines,
        ] table [
            x expr=\coordindex,
            y=pFB,
        ] {comparedef.txt};

    \addplot[draw=black,mark=*,thick,smooth] coordinates {(0,56.69) (1,51.92) (2,54.09) (3,52.90)};
    \addplot[draw=black,mark=*,thick,smooth] coordinates {(5,60.40) (6,51.83) (7,52.81) (8,51.85)};
    \addplot[draw=black,mark=*,thick,smooth] coordinates {(10,58.70) (11,53.33) (12,52.90) (13,53.17)};
    \addplot[draw=black,dashed,thick,smooth] coordinates {(0,50)(1,50)(2,50)(3,50)(4,50)(5,50)(6,50)(7,50)(8,50)(9,50)(10,50)(11,50)(12,50)(13,50)};


    % add the category labels
        \begin{scope}[
            % because the reference point will be the lower axis line the
            % labels have to be moved a bit more down to don't overlap with
            % the `xticklabels'
            every label/.append style={
                label distance=2ex,
            },
        ]
            \node [label=below:NiN] at (axis cs:1.5,\pgfkeysvalueof{/pgfplots/ymin}) {};
            \node [label=below:AlexNet] at (axis cs:6.5,\pgfkeysvalueof{/pgfplots/ymin}) {};
            \node [label=below:VGG] at (axis cs:11.5,\pgfkeysvalueof{/pgfplots/ymin}) {};

        \end{scope}
    \legend{Train Accuracy,Test Accuracy,Inference Accuracy}
    \end{axis}
\end{tikzpicture}
}
\caption{\underline{Comparison with prior defences.} Models trained using \method\hspace{0.02in} training methodology are comparable to prior state of the art defences: Adversarial Regularization and Differential Privacy, in terms of test accuracy and inference accuracy while additionally ensuring efficiency.}
\label{fig:compare}
\end{center}
\end{figure}


%plot gen error cdf


\subsubsection{Comparison with MemGuard}

The defences proposed so far can be categorized into (a) regularization based train-time defences and (b) post-training inference time defence.
Adversarial Regularization, Differential Privacy and other standard regularization techniques such as L2 and Dropout modify the training of the neural network.
Our proposed training framework exploits is part of category (a) where we modify the training of the machine learning model in order to provide acceptable levels of privacy and accuracy.
MemGuard~\cite{10.1145/3319535.3363201}, on the other hand, is a post-training defence, where the defender adds carefully crafted noise to the target model's output observations to ensure the misclassification of the adversary's attack classifier network.
The defence is based on the idea that the adversary's attack model is a machine learning classifier which is vulnerable to change in output with a carefully added noise to input (referred to as adversarial examples).
However, this post-training approach can be used in addition to the models trained using the \method\hspace{0.02in} framework.
Further, the attack that we use does not rely on an attack classification network but rather relies on output posterior to perform the attack.
Hence, within the threat model considered, the defence is not valid and we only compare our work with the state of the art defences that modify the training algorithm namely, Adversarial Regularization and Differential Privacy.

\section{Related Work}\label{related}

Data privacy in Machine Learning addresses different inference attacks such as membership inference in a blackbox setting~\cite{salem2018ml,shokri2017membership} or in the context of whitebox setting~\cite{DBLP:journals/corr/abs-1812-00910}.
Further, generative model have been shown to be vulnerable to membership inference attacks~\cite{LOGANMembershipInferenceAttacksAgainstGenerativeModels} and distributed setting such as in federated learning have also been exploited~\cite{melis2019exploiting,DBLP:journals/corr/abs-1812-00910}.
These privacy leakage in machine learning models have been mainly attributed to the memorization of training data by the models~\cite{236216,10.1145/3133956.3134077}.
In order to mitigate against inference attacks several defences have been explored such as Differential Privacy~\cite{Abadi:2016:DLD:2976749.2978318}, simple and adversarial regularization~\cite{DBLP:conf/ccs/NasrSH18,salem2018ml} which aim to generalize the model and alternatively, adding noise to the predictions to increase error~\cite{10.1145/3319535.3363201}.
Alternatively, confidential computing aims to privately and efficiently compute machine learning models using secure multiparty computation~\cite{235489}.
Interestingly, post-training approaches assuming the adversary uses shadow model attack (e.g., MemGuard~\cite{10.1145/3319535.3363201}), can exploit \method\hspace{0.02in} by using the models trained by this framework before to add noise.

Hardware-software co-design is crucial to accelerate the performance of NNs for embedded systems.
Hardware accelerators reuse weights and intermediate computation enable significant performance improvement~\cite{10.1109/ISCA.2016.30}.
Algorithmic optimizations have explored model compression through pruning~\cite{Han:2015:LBW:2969239.2969366} and reducing the precision of the model parameters and activations to binary~\cite{NIPS2016_6573}, ternary~\cite{Li2016TernaryWN} and generic quantization~\cite{Hubara:2017:QNN:3122009.3242044}.
Binarization enables to replace multiplication with simple boolean logic improving the overall performance~\cite{rastegari2016xnornet}.
Alternatively, hardware optimizations have enabled to design NN accelerators for low precision NNs for further efficiency~\cite{Umuroglu2017FINNAF}.
Further, specialized architectures designed for low memory footprint have also been extensively used for low powered devices such as mobile phones and micro-controllers~\cite{DBLP:journals/corr/IandolaMAHDK16,conf/cvpr/SandlerHZZC18}. 
However, all these optimization designs do not accounted for the resistance against inference attacks.
In this paper, we quantify the privacy leakage for different optimization and design algorithms for NNs and propose a training framework to reduce it.


%\noindent\textbf{Machine Learning Privacy.} Data privacy in Machine Learning addresses different inference attacks such as membership inference~\cite{salem2018ml,shokri2017membership}.

%\noindent\textbf{Efficient Deep Learning.} Hardware-Software Co-Design is crucial to accelerate the performance of NNs on hardware.


\section{Conclusions}\label{conclusions}

On device processing of sensitive data using Neural Networks on embedded systems requires a careful analysis of privacy, efficiency and accuracy of the algorithms which is currently lacking in literature.
In this work, we propose a two phase \method\hspace{0.02in} framework to design private, efficient and accurate NNs for execution on low powered embedded devices.
We quantify the privacy leakage using membership inference attacks where the adversary aims to infer whether a given data record was used in the model's training data.
We first provide a comprehensive privacy and efficiency analysis of state of the art algorithms for improving efficiency: model compression (pruning), quantization and efficient off-the-shelf architectures.
We show that model compression results in leaking more information compared to the original model while efficient off the shelf architectures do not provide the best efficiency guarantees.
Based on these observations, in Phase I of \method, we use quantization (specifically binarization) as a design choice which shows high resistance against inference attacks while satisfying all the efficiency requirements.
While Phase I optimizes for privacy and efficiency, in Phase II, we improve the accuracy of the resultant model using knowledge transfer from full precision (teacher) models to Phase I quantized model to improve the accuracy.
Our extensive evaluations on state of the art architectures on CIFAR10 dataset indicates that models trained using the proposed framework provides high resistance against membership inference attacks (comparable to other state of the art defences) with high efficiency against inference attacks.



%{\footnotesize
%\bibliographystyle{IEEEtranS}
%\bibliography{paper.bib}
%}

\bibliographystyle{ACM-Reference-Format}
\bibliography{paper}



\end{document}
\endinput
