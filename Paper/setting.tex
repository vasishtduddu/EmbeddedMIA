\section{Experiment Setting}


\subsection{Datasets}

\textbf{FashionMNIST.} This dataset consists of a training set of 60,000examples and a test set of 10,000 examples [63]. Each example isa28x28grayscale image, associated with a class label from 10fashion products, such as shirt, coat, sneaker.

\textbf{CIFAR10.} This dataset is composed of32x32color images in10 classes, with 6,000 images per class. In total, there are 50,000training images and 10,000 test images

\textbf{Purchase100.} The  Purchase100  dataset  contains  the  shop-ping  records  of  several  thousand  online  customers,  extractedduring  Kaggle’s  “acquire  valued  shopper”  challenge.2Thechallenge  was  designed  to  identify  offers  that  would  attractnew  shoppers.  We  used  the  processed  and  simplified  versionof  this  dataset  (courtesy  of  the  authors  of  [6]).  Each  recordin  the  dataset  is  the  shopping  history  of  a  single  user.  Thedataset  contains600different  products,  and  each  user  has  abinary record which indicates whether she has bought each ofthe  products  (a  total  of197,324data  records).  The  recordsare  clustered  into100classes  based  on  the  similarity  of  thepurchases,  and  our  objective  is  to  identify  the  class  of  eachuser’s purchases.
\footnote{https://kaggle.com/c/acquire-valued-shoppers-challenge/data}

\textbf{Location.} We created a location dataset from the publiclyavailable set of mobile users’ location “check-ins” in theFoursquare social network, restricted to the Bangkok areaand collected from April 2012 to September 2013 [36].7Thecheck-in dataset contains11,592users and119,744locations,for a total of1,136,481check-ins. We filtered out users withfewer than 25 check-ins and venues with fewer than 100 visits,which left us with5,010user profiles. For each location venue,we have the geographical position as well as its location type(e.g., Indian restaurant, fast food, etc.). The total number oflocation types is128. We partition the Bangkok map into areasof size0.5km0.5km, yielding318regions for which wehave at least one user check-in.Each record in the resulting dataset has446binary features,representing whether the user visited a certain region orlocation type, i.e., the user’s semantic and geographical profile.The classification task is similar to the purchase dataset. Wecluster the location dataset into30classes, each representinga different geosocial type. The classification task is to predictthe user’s geosocial type given his or her record. We use1,600randomly selected records to train the target model. \footnote{https://sites.google.com/site/yangdingqi/home/foursquare-dataset}

\subsection{Architecture and Hyperparameters}


\begin{table}[!htb]
\caption{Architectures for CIFAR10 and FashionMNIST datasets.}
\centering
\renewcommand\arraystretch{1.5}
\fontsize{6.7pt}{6.7pt}\selectfont
\resizebox{0.7\columnwidth}{!}{
\begin{tabular}{|c|c|c|}
\hline
\textbf{Purchase100} & \textbf{Location} & \textbf{FMNIST CNN}\\
\hline
Dense 1024 & Dense 512 & Convolution 32 (3,3)\\
Dense 512  & Dense 256 & Convolution 64 (3,3)\\
Dense 256  & Dense 128 & MaxPool (2, 2)\\
Dense 128  & Dense 30  & Dense 128 \\
Dense 100  &           & Dense 10\\
           &           & \textbf{FMNIST MLP}\\
           &           & Dense 512\\
           &           & Dense 512\\
           &           & Dense 512\\
\hline
\end{tabular}
}
\label{tab:architectures}
\end{table}
