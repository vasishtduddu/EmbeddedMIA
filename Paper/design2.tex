\section{\method: Design Overview}\label{design}

%Based on the comparative analysis described in the previous section
In this section we detail \method\hspace{0.02in}.
\method~ is a technique to construct NNs dedicated to embedded systems with efficiency, accuracy and privacy as main requirements:

\begin{itemize}[leftmargin=*]

\item {\em Privacy.}
The model should preserve the privacy of an individual's data record in the training set of the model against inference attacks.

\item {\em Efficiency.}
The private model should demonstrate high energy, memory and computation efficiency for practical deployment to embedded and mobile devices.

\item {\em Accuracy.}
The accuracy degradation of the private and efficient model should be minimum as compared to the non-private and non-efficient model accuracy.
\end{itemize}


%Based on the efficiency and privacy analysis described in the previous section, we describe the detailed \method\hspace{0.02in} framework for designing efficient, private and accurate NNs in this section.
\method~ is composed of two phases.
In Phase I, the objective is to enhance the model's efficiency and privacy, however, at the cost of accuracy (Section~\ref{p1}).
In Phase II, we optimize for accuracy and train the resultant model from Phase I (Section~\ref{p2}).


%\subsection{Overall Design Requirements}

%In designing our training methodology, \method, we aim to satisfy the following main requirements in the purview of embedded systems:


%To this end, we present \method --- a technique to construct NNs that are optimized for efficiency and accuracy while ensuring privacy of the input data.

%With \method, we present a novel approach of building NNs with efficiency as a key property.
%We argue that fixing a NN architecture and then modifying the training algorithm to ensure privacy (e.g. adversarial regularization[] and differential privacy[]) does not provide optimal balance between efficiency and accuracy.
%On the contrary, our approach advocates building NN architecture considering the strengths and drawbacks of the underlying algorithms to design efficient NNs.
%We assert that this is a practical solution as deep learning algorithms are flexible with respect to their architectures, i.e., different NNs can be trained to achieve the same accuracy for a given dataset.



\subsection{Phase I}
\label{p1}

We first quantize the model's parameters and intermediate activations. We specifically binarize the values to restrict them to the values of \{+1,-1\}.
This operation (as seen in Section~\ref{quant}) results in high resistance to inference attacks as well as satisfies the different efficiency requirements.
The model achieves computation efficiency by replacing the expensive matrix multiplications with simple boolean arithmetic operations, i.e, XNOR computations.
Alternatively, instead of using multiplication and addition circuits in the hardware, we leverage XNOR logic on the inputs followed by a bitcount operation (counting the number of high bits "1" in a binary output sequence).
The equation can be represented as follows:

\begin{align}
\footnotesize
\mathbf{x} \cdot \mathbf{w} =
N - 2\times\operatorname{bitcount}(\operatorname{xnor}(\mathbf{x}, \mathbf{w}))
\end{align}

In terms of memory efficiency, binarization results in a direct reduction of the model size as well as intermediate output memory requirements by 32x to 64x.
Lowering the precision also reduces the number of memory access by 32x to 64x resulting in a significant decrease in the total energy consumption.


\begin{algorithm}
\footnotesize
\begin{algorithmic}
    \FOR{$k=1$ to $L$}
        \STATE $W_k^b \leftarrow {\rm Binarize}(W_k)$
        \STATE $a_k \leftarrow N - 2\times\operatorname{bitcount}(\operatorname{xnor}(\mathbf{a_{k-1}^b}, \mathbf{W_k^b}))$
        \IF{$k < L$}
            \STATE $a_k^b \leftarrow {\rm Binarize}(a_k)$
        \ENDIF
    \ENDFOR
\end{algorithmic}
\caption{Inference Stage of Binary Neural Network with XNOR Operations where $W_k^b$ are the binarized weights($W_k$) and $a_k$ is the activation of the $k^{th}$ layer}
\label{alg:inference}
\end{algorithm}


The complete inference stage of the Binarized NN with XNOR computation is given in Algorithm~\ref{alg:inference}.
The matrix multiplication between the previous layer activation $a_{k-1}$ and the current layer's weights with the bitcount of XNOR operation's output.
The function Binarize() is a deterministic thresholding function which maps the input values to the set \{-1,+1\}.
In addition to the above design, we use additional optimizations for XNOR-Net to avoid a significant loss in accuracy.
It is well documented that it is difficult to converge a binarized model during training~\cite{AAAI1714619} in case of incompatible hyperparameter settings. To this extent, we use the first and last layer of the model as full precision.
These additional optimizations have been used previously for XNOR based networks~\cite{8114708,rastegari2016xnornet} and provides higher accuracy and model convergence at a small cost of memory and energy consumption overhead.







\subsection{Phase II}
\label{p2}

In Phase I, we optimize for both privacy and efficiency but these are achieved at the significant degradation of accuracy (Section~\ref{evalPh1}).
In order to restore the accuracy close to the original full precision accuracy, we propose to use knowledge distillation~\cite{44873}.
Here, we consider a pre-trained teacher model $f_{teacher}()$ with state of the art accuracy on the classification task and use it to guide the training of the quantized classifier.
During training of the quantized model (student), we do not compute the loss between the true label $y$ and predicted label $f_{student}(x)$.
We instead estimate the loss between the predicted label $f_{student}(x)$ and the predicted label for the full precision teacher model $f_{teacher}(x)$.
The loss function in knowledge distillation $Loss_{KD} (f_{student}, f_{teacher})$ is given as

\begin{equation}
\footnotesize
\sum_{k=1}^m f_{student}(x_k)log(f_{teacher}(x_k)) + f_{teacher}(x_k)log(f_{student}(x_k))
\end{equation}


This ensures that the student model learns to map the prediction boundary of the teacher model and imitates the prediction behaviour for different inputs.
This results in an increase in the accuracy of the student model compared to the original baseline of standalone training without the teacher model.
